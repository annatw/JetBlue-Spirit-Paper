    
	% This section is organized into three subsections. I estimate a random-coefficient nested logit model of demand (Section \ref{sec:Analysis_Demand}), assume Bertrand-Nash competition with differentiated products to estimate marginal costs (Section \ref{sec:Analysis_Supply}), and then estimate three merger counterfactual simulations which are roughly a best case, an average case, and a worst case for the merger's effects on consumers (Section \ref{sec:Analysis_Supply}). 
	
	% Before beginning these analyses, it is important to consider the implications of one critical assumption used within the simulation, namely that firms treat the overall market structure as exogenously determined when determining prices.\footnote{This is consistent with the internal pricing processes of airlines, discussed in \citet{hortacsu_organizational_2024}.} The most immediate consequence of this is that routing within the models discussed in this paper cannot respond to demand shocks within a given quarter. Furthermore, within the simulations, firms do not change which markets they operate in following the merger nor reposition their products within these markets except through price and changing the branding. % Two recent papers relax this assumption. In \citet{ciliberto_market_2021}, Meanwhile, \citet{li_repositioning_2022} estimates a static equilibrium model where firms can reposition their products' type (connecting or nonstop) in response to a merger. Their model predicts limited entry of new, nonstop products following three historical mergers. 
	
	% This creates a problem for proper inference of the counterfactual world in which the merger had been completed. The NEA between American and JetBlue had included a reorganization of the route networks of each of the collaborating firms as part of the agreement. Furthermore, it allowed for codesharing between the airlines.\footnote{One consequence of this is that NEA codeshare products ticketed to JetBlue are included as JetBlue products for purposes of the merger simulation.} As such, counterfactuals using markets between 2021 and 2023 are those in which the JetBlue-Spirit merger was completed while the NEA is in effect and without any resulting reorganization of routes. However, as the NEA would ultimately be ruled against before the beginning of the trial for the JetBlue-Spirit merger, it is hard to believe in the existence of this world had the JetBlue-Spirit merger been allowed to be completed.
	
	% Meanwhile, simply using a sample from before the implementation of the NEA would require the usage of data from before the coronavirus pandemic. As documented in press sources and in a working paper \citep{ewen_zoom_2023}, air travel demand dynamics have greatly changed following the pandemic. In part thanks to the rise of telecommunications software such as Zoom, low-price elasticity business travel has lessened. Concurrently, American consumers acquired additional savings during the pandemic which they were able to use on additional consumption following the pandemic \citep{klitgaard_spending_2023}. As such, the change in the price elasticity of air travel following the pandemic is a priori ambiguous.  
	
 	% To mitigate these issues, I conduct all analyses within this section on two samples rather than one. The first of these is a "pre-pandemic" sample which includes markets from the first quarter of 2017 through the fourth quarter of  2019 and a "post-pandemic" sample consisting of markets from the second quarter of 2021 through the end of the second quarter of 2023.\footnote{The second quarter of 2021 is the first quarter following the coronavirus pandemic with widespread vaccine availability. The second quarter of 2023 is the last quarter with the NEA completely in effect, as such, having this as the terminating date preserves clarity of the estimated counterfactual.} By comparing the results from these two samples, a more complete picture can emerge of the counterfactual world in which the JetBlue-Spirit merger had been completed despite the issues facing each sample's overall credibility.  