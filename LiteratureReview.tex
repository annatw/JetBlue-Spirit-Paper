\documentclass{article}
\usepackage{setspace}
\usepackage{natbib}


\begin{document}
    
    \section{Literature Review}
	\label{sec:Literature}

    This paper analyzes the counterfactual effects of the proposed JetBlue-Spirit merger. To accomplish this, it draws from and contributes to the economic literature analyzing the merger-induced consolidation within the aviation industry during the twenty-first century.  It additionally furthers the understanding of the role of low-cost carriers and ultra-low cost carriers within the aviation industry in promoting competition through the analysis of more recent data which includes the period following the COVID-19 pandemic.
	
	Since the turn of the century, mergers have led to increased consolidation within the aviation industry. The existing literature is divided on the net price effects of these mergers, due to differences of merger characteristics and methodology. To understand the impact of different methodologies on the estimated pro- or anti-competitive price effects, consider \citet{luo_price_2014} and \citet{carlton_are_2019}. These papers estimated pro-competitive effects of the Delta-Northwest, United-Continental, and American-US Airways mergers by using a differences-in-differences inference design which did not allow for dynamic realization of effects and instead relied on the determination of pre- and post-period. This methodology would be challenged in \citet{fan_when_2020} which found evidence for price increases following the United-Continental merger through the adoption of a model which allowed for dynamic price effects. This found evidence consistent with prices rising following merger announcements rather than completion, which reduced the estimated effects in the prior literature. Research, such as \citet{bet_retrospective_2021, ciliberto_market_2021}, has additionally been conducted using structural modeling to recover firm marginal costs and markups to analyze the effects of these mergers. \citet{bet_retrospective_2021} found evidence for increases in markups resulting from the United-Continental, Southwest-AirTran, and American-US Airways mergers but not the Delta-Northwest merger while estimating limited efficiency gains across all of these mergers. \citet{ciliberto_market_2021} estimated the effects of the American-US Airways merger using a structural model which allowed for dynamic entry into markets, and estimated price increases of around 5\% for duopoly markets consolidated by the merger. Within this paper, structural modeling is used to assess the counterfactual effects of the JetBlue-Spirit merger. 

    In contrast to these papers focused on mergers of legacy carriers, this paper focuses on a proposed (but ultimately uncompleted) merger between a low-cost and an ultra-low cost carrier. As part of this analysis, it simulates the merger of these two airlines and finds evidence of heterogeneous fare increases following the merger. Recently, \citet{ciliberto_market_2021} and \citet{li_repositioning_2022} have used simulations of legacy carrier mergers to examine the implications of models that account for route entry decisions. I discuss the ramifications of these papers on the assumption of exogenous market structure for this paper in detail in Section \ref{sec:Analysis}. 

    Past research has identified strong competitive effects from low-cost carriers, such as Southwest, on incumbent firms within a market \citep{morrison_actual_2001, goolsbee_how_2008}. Spirit in particular has been linked to increased fare dispersion following its entry into a given route through inducing competing airlines to introduce 'basic economy fares'. This is associated with higher declines in price for cheaper fares than for more expensive fares, which Spirit less directly competes against \citep{shrago_spirit_2024}.

    % Above - Introduce Delta Anecdote?
    
	Finally, my paper touches upon the literature examining the role of low-cost and ultra-low cost carriers within the aviation industry. While this literature has predominantly focused on the ability of Southwest, the largest low-cost carrier, to lower prices within a market (e.g. \citet{windle_short_1995, morrison_actual_2001,  goolsbee_how_2008}), there has been movement in recent years to examine the effects of other low-cost carriers, such as JetBlue and Spirit, on prices. One example of this strand of literature is \citet{shrago_spirit_2024} which found that Spirit entry into a market was responsible for lower prices for the cheapest fares but minimal effect on average fares. In addition to my paper's analysis of the pricing effects of an unrealized merger between an ultra-low cost and low-cost carrier, it reports stylized facts about the changing role of low-cost and ultra-low cost carriers in the late 2010s and early 2020s.  

\end{document}