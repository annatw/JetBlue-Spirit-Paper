\documentclass{article}
\usepackage{setspace}
\usepackage{natbib}
\usepackage{amsmath}
\usepackage{float}
\usepackage{longtable}
\usepackage{booktabs}
\usepackage{lscape}
\usepackage{graphicx}
\usepackage{silence}
\usepackage{forest}
\usepackage{hyperref}
\usepackage{placeins}
\usepackage{textcomp} % Needed on Windows in Office
\usepackage{adjustbox}
\usepackage{xcolor}
\usepackage[letterpaper, margin=1in]{geometry}

\usepackage[toc,page]{appendix}


\let\Oldsubsection\subsection
% \renewcommand{\subsection}{\FloatBarrier\Oldsubsection}
\newcommand{\comment}[1]{}

\author{Ann Atwater\footnote{Department of Economics, University of Florida.}}
\title{Maverick Firms And Competition: Evidence from the Attempted JetBlue-Spirit Merger \footnote{This is a preliminary draft. Please do not distribute without permission of the author. The author would like to thank the participants of the University of Florida Applied Microeconomics working group and of the Southern Economics Association 2024 conference for helpful feedback. Furthermore, the author would like to thank Brad Shrago for alerting her to the reliance on usage fees by ultra-low cost carriers. }}

\graphicspath{{05.Figures/}}

\begin{document}
	\maketitle
	
	\begin{abstract}
In 2024, the attempted merger between JetBlue Airways and Spirit Airlines was blocked following a lawsuit brought by the Department of Justice. This paper estimates the counterfactual effects from this merger and finds evidence consistent with gains for the average consumer of air travel but harms for extremely price-constrained consumers, consistent with the judgment in the case. Furthermore, I estimate significant heterogeneity in within-market effects - even under the specification most favorable to the merger, I estimate that over 35 markets would have had their minimum fares increase by over \$60. However, proper counterfactual evaluation is contaminated by changes in the route structure of JetBlue caused by its Northeast Alliance with American Airlines. To resolve this, I estimate the impacts of the Northeast Alliance on JetBlue's product choices and market fares. I find that the Northeast Alliance increased fares at Boston Logan International Airport while decreasing them at Newark Liberty International Airport. \bigskip
	%To account for Spirit's bankruptcy, I estimate one final set of counterfactuals to simulate the world wherein Spirit leaves the aviation market. 	
	\noindent JEL Classification: L4, L41 \newline
	\noindent Keywords: airlines; mergers
		
	\end{abstract}
	
	\pagebreak
	
	\doublespacing
	
	\section{Introduction}
	\label{sec:Introduction} 

    In 2024, a proposed merger between JetBlue and Spirit became the first merger within the American aviation sector to be blocked following trial. This merger would have seen the largest ultra-low cost carrier acquired by a low-cost carrier. Spirit had, over the course of the 2010s, pioneered the ultra-low cost carrier business model and acted as a maverick firm which competed vigorously for budget conscious travelers. As such, understanding the counterfactual effects of this merger allows for an increased understanding of the role of maverick firms in promoting competition. 
 
    The estimation of the effects of this merger is complicated by two factors - JetBlue's Northeast Alliance (NEA) with American Airlines and the COVID-19 pandemic. Under the NEA, JetBlue and American Airlines coordinated on setting routes to and from airports serving the Boston and New York City metropolitan areas (Boston Logan International Airport, Newark Liberty International Airport, John F. Kennedy International Airport, and LaGuardia Airport) between 2020 and 2023. This agreement would be found to be anti-competitive following trial in 2023. Due to this agreement, JetBlue operated in different markets than it otherwise would have, inhibiting the understanding of the counterfactual world where the merger had been approved.

    Concurrent with this anti-competitive conduct, the aviation industry was dealing with the ramifications of the COVID-19 pandemic on consumption patterns. While air travel levels would return to 2016 levels by the second quarter of 2021, less price-conscious business travel had dramatically decreased due to the increased prevalence of telecommunications use for meetings and has only recently recovered to pre-pandemic patterns. Leisure travelers, traditionally more price sensitive than business travelers, were observed to be behaving less price sensitively than normal by airlines, possibly owing to acquired savings during the travel slowdown.

    To resolve the issues presented by these factors, I estimate the counterfactual price effects of the merger on markets within the three years before the pandemic and the three years after the second quarter of 2021. While each period of analysis is separately flawed (the pre-pandemic period has normal routing but inaccurate demand patterns while the post-pandemic period has adversely impacted routing with normal demand patterns), the net picture of the merger is clear. I estimate anti-competitive price increases dominating fares at the lower end of the fare distribution, with estimated minimum fare increases of over \$80 following the merger in over thirty-five markets under the assumptions most favorable to the merger's efficiencies, regardless of the sample period used. This contrasts with the mixed picture of the merger's effects on average fares, with average fares estimated to not change in the pre-pandemic markets and increasing on average by about 4\% in the post-pandemic markets. 

    The difference between these estimated effects for the average and minimum market fares speaks to the degree that Spirit's competitive impacts focused on the price constrained travelers. In line with previous research, Spirit's competitive effects are strongest at the low-end of the fare distribution and so careful analysis of the heterogeneity across markets must be undertaken to properly understand the effects of its existence. 
    
	With these findings outlined, the rest of the paper is organized as follows: Section \ref{sec:Literature} briefly summarizes the literature on airline mergers and role of low-cost carriers and ultra-low cost carriers within the industry; Section \ref{sec:Setting} details on the American air travel market, the Northeast Alliance, and the JetBlue-Spirit merger; Section \ref{sec:Data} elaborates on my data sources and presents summary statistics; Section \ref{sec:Analysis} contains my analysis of the Northeast Alliance and the JetBlue-Spirit merger; finally, Section \ref{sec:Conclusion} summarizes the findings of this paper and examines their implications for antitrust policy going forwards. 
  
	\section{Literature Review}
	\label{sec:Literature}

    This paper analyzes the impact of the JetBlue-Spirit merger in the world in which the merger had been completed and not blocked at trial. To accomplish this, it primarily draws from the literature analyzing the merger-induced consolidation during the twenty-first century.  Beyond this, it furthers the understanding of the role of low-cost carriers and ultra-low cost carriers within the aviation industry in promoting competition through the analysis of more recent data which includes the changes in the industry following the COVID-19 pandemic.
	
	Since the turn of the century, the aviation industry has seen a trend toward consolidation of existing carriers through mergers. The existing literature is largely divided on the net price effects of mergers within this sector, owing to differences in the various settings of the mergers as well as in methodology. To understand the impact of different methodologies on the estimated pro- or anti-competitive price effects, consider \citet{luo_price_2014} and \citet{carlton_are_2019}. These papers estimated pro-competitive effects of the Delta-Northwest, United-Continental, and American-US Airways mergers by using a differences-in-differences inference design which did not allow for dynamic realization of effects and instead relied on the determination of pre- and post-period. This methodology would be challenged by \citet{fan_when_2020} which found evidence for price increases following the United-Continental merger while using a differences-in-differences design which allowed for price effects to vary quarterly as prices began to rise following the announcement of the merger rather than its completion. Beyond papers using differences-in-differences inference design, research has been conducted using structural modeling to recover firm marginal costs and markups. \citet{bet_retrospective_2021} found evidence for increases in markups resulting from the United-Continental, Southwest-AirTran, and American-US Airways mergers but not the Delta-Northwest merger while estimating limited efficiency gains across all of these mergers. \citet{ciliberto_market_2021} estimated the effects of the American-US Airways merger using a structural model which allowed for dynamic entry into markets, and estimated price increases of around 5\% for duopoly markets consolidated by the merger. Within this paper, structural modeling is used to assess the counterfactual effects of the JetBlue-Spirit merger. 

    One consideration of the attempted JetBlue-Spirit merger that is unique to it and not shared with these prior mergers is that it would involve the merging of an ultra-low cost carrier with a low-cost carrier. Past research has identified strong competitive effects from low-cost carriers, such as Southwest, on incumbent firms within a market \citep{morrison_actual_2001, goolsbee_how_2008}. Spirit in particular has been linked to increases in fare dispersion of existing airlines through the introduction of 'basic economy fares'. This has resulted in large reductions in fares for the cheapest fares on routes that it operates in \citep{shrago_spirit_2024}.
    
    In contrast to these papers focused on mergers of legacy carriers, this paper focuses on a proposed (but ultimately uncompleted) merger between a low-cost and an ultra-low cost carrier. As part of this analysis, it conducts a simulation of the merger of these two airlines. Recently, \citet{ciliberto_market_2021} and \citet{li_repositioning_2022} have used simulations of legacy carrier mergers to examine the implications of models that account for route entry decisions. I discuss the ramifications of these papers on the assumption of exogenous market structure for this paper in detail in Section \ref{sec:Analysis}. 
	
	Finally, my paper touches upon the literature examining the role of low-cost and ultra-low cost carriers within the aviation industry. While this literature has predominantly focused on the ability of Southwest, the largest low-cost carrier, to lower prices within a market (e.g. \citet{windle_short_1995, morrison_actual_2001,  goolsbee_how_2008}), there has been movement in recent years to examine the effects of other low-cost carriers, such as JetBlue and Spirit, on prices, such as the aforementioned \citet{shrago_spirit_2024}. In addition to this paper's analysis of the pricing effects of an unrealized merger between an ultra-low cost and low-cost carrier, it reports stylized facts about the changing role of low-cost and ultra-low cost carriers in the late 2010s and early 2020s.  
	
	\section{Setting}
	\label{sec:Setting}
	
	Within this section, I establish key facts about the aviation industry in the United States in Subsection \ref{sec:Setting_Aviation}, then I discuss the Northeast Alliance and JetBlue-Spirit merger in more detail in Subsections \ref{sec:Setting_NEA} and \ref{sec:Setting_Merger} respectively.
	
	\subsection{United States Aviation}
	\label{sec:Setting_Aviation}
	The American aviation industry is primarily comprised of three types of carriers: legacy carriers, low-cost carriers, and ultra-low cost carriers. Beyond pricing differences, each category of firms is defined by differences in their operations. As such, it is worth spending time on these differences, as they inform later analyses within this paper. % May cut word "major" here - I don't discuss airlines which operate codeshares with larger firms.
	
	Legacy carriers are those firms that operated in the industry before fare deregulation in 1978. Following a series of mergers in the last few decades, Delta, American, and United are the only legacy carriers still operating today. These carriers operate hub-and-spoke route networks which require customers to connect through a centralized hub to reach smaller destinations. One consequence of this is that these firms operate a larger variety of aircraft within their fleets than firms of the other two types. This allows for the servicing of smaller markets with minimal excess capacity at the cost of additional crew training and maintenance expenditures.   
	
	Non-legacy air carriers are divided into two groups, low-cost carriers and ultra-low cost carriers. Low-cost carriers include Southwest and JetBlue, while the ultra-low cost carriers are comprised of Spirit, Allegiant, and Frontier.\footnote{Alaska Airlines and Hawaiian Airlines are larger carriers with a regional focus. They both operate a model closer to legacy carriers than those of low-cost carriers. Furthermore, there exist several smaller, more regional-focused low-cost carriers, such as Sun Country. Later analyses within this paper treat products from these airlines as from a unified, "Minor Low-Cost Carrier" airline.} Unlike legacy carriers, both low-cost and ultra-low cost carriers favor the usage of direct flights. While this requires them to eschew smaller markets, it allows for the avoidance of expenditures relating to operating hub airports. Furthermore, this focus on direct flights allows these carriers to use only a limited number of aircraft models within their fleets.

	Ultra-low cost carriers are distinguished from low-cost carriers through the practice of ``unbundling," wherein ticket prices are lower but amenities traditionally included in a fare are additional purchases. While Ryanair in Europe has operated under this model since the 1990s, a United States firm would not successfully adopt the strategy until Spirit introduced fees for checked baggage in 2010 \citep{bachwich_emergence_2017}. While complaints regarding the quality of these airlines are well documented in consumer surveys and the press \citep{vasel_spirit_2016, elliott_jetblue_2022}, these airlines have managed a good deal of competitive success within their segment of the market by targeting highly budget-conscious travelers who do not wish to pay more expensive fares. By the later part of the 2010s, trips on ultra-low cost carriers represented over a tenth of total air travel within the United States.  Despite this growth, the industry is still dominated by the ``big four" carriers - the three legacy carriers who along with Southwest comprise approximately three-quarters of the overall passenger trips within the United States. 

     Through adopting the ultra-low cost carrier model, Spirit was able to rapidly grow within the post-recession domestic aviation landscape. As documented in Figure \ref{fig:Both_fleet}, Spirit grew its fleet from under 50 planes before adopting the ultra-low cost model to nearly 200 planes by 2022. As part of this expansion in operations, Spirit increasingly competed against JetBlue (Figure \ref{fig:JBSpirit_Airports_2022}).\footnote{This is especially notable as very few markets have historically had multiple low-cost carriers operating within them \citep{kwoka_fringe_2016, ciliberto_market_2021}. However, in recent years, there are approximately half as many markets with multiple low-cost carriers operating within them as there are with only a single low-cost carrier.} Both firms primarily operate in airports situated along the eastern seaboard of the United States in addition to Las Vegas and major cities in Texas and California.
	
	Despite these similarities in operations, both firms behaved differently regarding competition by the 2020s.  In 2020, JetBlue created the ``Northeast Alliance" (NEA) with American Airlines, which saw cooperation on flights originating from or departing to airports within the New York City and Boston areas: LaGuardia Airport (LGA), John F. Kennedy International Airport (JFK), Newark Liberty International Airport (EWR), and Boston Logan International Airport (BOS).\footnote{Section \ref{sec:Setting_NEA} elaborates on this in more detail.} Beyond the NEA, the Department of Justice has alleged that JetBlue had taken part in anti-competitive behavior using the Airline Tariff Publishing Company to coordinate fares with other firms. 
	
	Spirit, on the other hand, competed aggressively and operated as  a maverick firm within the aviation industry. It has maintained a consistent pace of increasing its fleet size throughout the 2010s and into the 2020s (as graphed in Figure \ref{fig:Both_fleet}) despite the shock to air travel caused by the coronavirus pandemic.\footnote{As depicted in the aforementioned figure, JetBlue's fleet stagnated following the pandemic due to it negotiating delayed fulfillment of orders placed before the pandemic \citep{bellamy_iii_jetblue_2020, sipinski_jetblue_2020}.} This growth in its fleet was required for expansion into new markets. As shown in \citet{shrago_spirit_2024}, Spirit causes increased variance in fares within markets that it enters by inducing legacy carriers to compete for the highly cost-concerned travelers targetted by Spirit. They do this by offering "basic economy" fares which operate similarly to Spirit's ``unbundled'' fares.
	
	Finally, it is important to note the impact of the coronavirus pandemic on the aviation industry.  A severe drop in air travel occurred almost immediately as consumers and businesses canceled travel plans due to both virus concerns and government mandates. While widespread vaccine availability allowed for recovery to 2016 levels of air travel by the second quarter of 2021, passenger levels would not recover to 2019 levels of air travel until halfway into 2022 (see Figure \ref{fig:QuarterlyPass}). 

\begin{figure}
	\caption{Quarterly Passengers, All Carriers}
	\label{fig:QuarterlyPass}
	\includegraphics[width = \linewidth]{Quarterly_DB1B_Itineraries}
	\footnotesize{Source: DB1B Data. Shaded region depicts the duration of the coronavirus pandemic before widespread vaccine availability within the United States, namely, from the first quarter of 2020 through the first quarter of 2021.}
\end{figure}
    
	However, this recovery in ridership did not mark a return to the prior market conditions for the industry. Historically, approximately a third of air travel is motivated by business \citep{berry_tracing_2010, bet_market_2021}). However, following the pandemic, business travel decreased as businesses switched to telecommunications for meetings rather than face-to-face interactions \citep{semuels_business_2021}. Meanwhile, leisure travelers built savings during the decline in travel, allowing them to behave less price-sensitively following the pandemic. As such, the overall change in price elasticity is apriori ambiguous.\footnote{In a recent working paper, \citet{ewen_zoom_2023}, this phenomenon appears to have occurred, with a non-negligible share of leisure travelers being less price-sensitively than before the pandemic. In Table \ref{tab:DemandEst}, I find evidence that airfare has become slightly less elastic following the pandemic, consistent with the idea that this change in consumer behavior was enough to offset the changes from business travel on elasticity.}  As such, consumption patterns are liable to differ between the pre-pandemic and post-pandemic periods despite the large recovery in passenger levels. 
	
	\subsection{Northeast Alliance}
	\label{sec:Setting_NEA}
	
	Prior to its attempted merger with Spirit, JetBlue entered into the Northeast Alliance (NEA) with American Airlines at the start of 2021. The NEA saw the two firms coordinate operations to behave as if they were in fact a single carrier for routes that touched upon the airports serving New York City and Boston. They jointly decided their network for routes, and operated them so that consumers would be indifferent between the two carriers for these routes\footnote{With the ultimate goal of keeping only one of the two firms active on a given segment, as per the ruling by Judge Leo Sorokin.}, and shared revenue between the firms earned from products within the agreement. The United States Department of Justice along with six states and the District of Columbia brought a lawsuit against the agreement in September 2021 alleging violations of the Sherman Antitrust Act. Following a 2022 trial, the agreement would be found to violate the Sherman Antitrust Act in May 2023 and it was subsequently unwound\citep{rennison_jetblue-american_2023, rains_what_2023}.\footnote{Table \ref{tab:NEA_Timeline} details a timeline of key events relating to the NEA. Notably, some landing slot leases between the two airlines related to the merger experienced a gradual reversion to their original owners.} With this timeline outlined, I will now briefly discuss the features of the NEA, how it differs from traditional aviation alliances, and the issues it posses for estimation of the counterfactual merger effects for the JetBlue-Spirit merger. 

    \begin{table}[tb]
		\caption{Northeast Alliance Timeline}
		\label{tab:NEA_Timeline}
		\begin{center}
			\begin{tabular}{ccc}
				\hline
				Year & Date & Event \\
				\hline
				2020 & Quarter 1-2 & JetBlue and American Negotiate Alliance \\ 
				& July 16 & Northeast Alliance Announced \\
				& July 22 & Alliance Agreement submitted to DOT \\
				\hline 
				2021 & January 10 & DOT Terminates Antitrust Review \\
				& February 24 & Codesharing Agreement Begins on {X} Routes \\
				& May 26 & Reciprocal Loyalty Earnings Begins \\
				& Early September & NEA Shuttle at JFK Opens \\
				& September 21 & DOJ Files Lawsuit Against NEA \\  
				\hline
				2022 & September 27 - November 18 & NEA-Trial \\
				\hline 
				2023 & May 19 & NEA Ruled Anti-Competitive \\
				& July 5 & JetBlue Drops Appeal Plans \\
				& July 21 & NEA Codesharing Ends \\
				& October 31 & JFK Shuttle Ceases Operation\\
				& October 31 & 12 Slot Leases to JetBlue Terminate \\
				\hline 
				2024 &  March 31  & 27 Slot Leases to JetBlue Terminate \\ 
				& March 31 & 1 Slot Lease to American Terminates \\
				& October 26 & Remaining NEA Slot Leases Terminate				 \end{tabular}
		\end{center}
	\end{table}

	
	Alliances between aviation firms are commonplace within consumer aviation. For example, Delta is part of the ``SkyTeam Alliance" and United is part of the ``Star Alliance." These alliances see firms operating code-sharing agreements between different carriers, in which airlines can sell seats on flights operated by other airlines, with the ticket being under the ticketing airlines' code.\footnote{This allows, for example, for passengers of the Canadian airline Westjet to book connecting flights into the United States by using Delta flights for the intranational legs of their trip.}  These agreements are primarily operated between carriers situated in different countries which allows for better access to foreign markets as airlines are not normally able to fly routes entirely situated within a single foreign country. Benefits to consumers include the earning of frequent flier miles across all stages of a journey, regardless of the operating carrier, easier handling of baggage, and easier bookings. Meanwhile, carriers benefit from being able to offer a wider variety of destinations than would otherwise be possible.
	
	Domestic aviation alliances are rare at present. Unlike alliances between domestic and foreign carriers, these agreements are generally unable to receive waivers from antitrust concerns through the Department of Justice and as such face additional regulatory issues. Traditionally, these domestic agreements only apply to markets without both firms and the alliance's members otherwise maintain separate routing decisions, operations, and planning. 
	
	In contrast, the NEA was structured to act as if it were a merger between the two firms on affected routes. The two firms jointly scheduled flights within the selected cities, to minimize overlap on routes operated by both firms\footnote{It is not clear how effective this was. As documented in Figure \ref{fig:NEA_Operating}, the levels of shared routes at each of the four impacted airports was within historically normal levels.}, and coordinated operations at the airports impacted by the agreement.\footnote{One example of this coordination is the shuttle operated by the two airlines at JFK to allow customers to transfer between the terminals used by each airline without having to clear security a second time on a connecting trip \citep{griff_riding_2021}.} Furthermore, as two of the impacted airports featured slot and gate controls,\footnote{A slot-controlled airport is one in which airlines are assigned specified time slots by the FAA for departures and arrivals to allow for better coordination of runway usage in congested airports. These slots are set in advance of individual operation days and can be transferred between airlines as if they were property.} the NEA saw these firms share slot permits and coordinate on sharing gates at the affected airports. This would be found by the trial court to have increased barriers to entering the New York City air travel market by deterring these firms by selling off landing slots to other firms. Finally, the two firms shared revenue from these markets to align their incentives with those of the NEA. 

    \begin{table}[h]
		\caption{American, JetBlue Overlap at NEA Airports}
		\label{tab:NEA_Airport_Prescence}
        \vspace{-15mm}
        \begin{center}
         
\begin{tabular}[t]{rrrrrrrrr}
\toprule
\multicolumn{1}{c}{ } & \multicolumn{2}{c}{JFK} & \multicolumn{2}{c}{BOS} & \multicolumn{2}{c}{LGA} & \multicolumn{2}{c}{EWR} \\
\cmidrule(l{3pt}r{3pt}){2-3} \cmidrule(l{3pt}r{3pt}){4-5} \cmidrule(l{3pt}r{3pt}){6-7} \cmidrule(l{3pt}r{3pt}){8-9}
Year & Ticket & Operating & Ticket & Operating & Ticket & Operating & Ticket & Operating\\
\midrule
\addlinespace[0.3em]
\multicolumn{9}{l}{\textbf{Q1}}\\
\hspace{1em}2023 & 75.4 & 23.7 & 69.1 & 21.4 & 67.3 & 4.2 & 46.7 & 7.7\\
\hspace{1em}2022 & 77.0 & 29.7 & 75.0 & 29.1 & 73.9 & 8.9 & 47.6 & 8.3\\
\hspace{1em}2021 & 18.6 & 24.4 & 26.8 & 21.7 & 50.0 & 33.3 & 9.5 & 13.6\\
\hspace{1em}2019 & 22.4 & 23.3 & 22.9 & 20.0 & 8.1 & 6.7 & 0.0 & 0.0\\
\addlinespace[0.3em]
\multicolumn{9}{l}{\textbf{Q2}}\\
\hspace{1em}2023 & 70.0 & 15.9 & 63.1 & 22.2 & 68.3 & 5.4 & 46.7 & 7.7\\
\hspace{1em}2022 & 68.3 & 26.5 & 70.3 & 21.9 & 75.0 & 6.4 & 45.5 & 17.4\\
\hspace{1em}2021 & 57.7 & 21.1 & 57.4 & 28.6 & 27.3 & 7.4 & 28.6 & 13.6\\
\hspace{1em}2019 & 21.1 & 21.0 & 21.2 & 25.0 & 8.9 & 5.6 & 0.0 & 0.0\\
\addlinespace[0.3em]
\multicolumn{9}{l}{\textbf{Q3}}\\
\hspace{1em}2023 & 69.0 & 16.9 & 63.9 & 19.0 & 61.0 & 4.1 & 40.0 & 7.7\\
\hspace{1em}2022 & 73.8 & 21.9 & 73.8 & 24.2 & 75.9 & 6.4 & 57.1 & 7.1\\
\hspace{1em}2021 & 63.6 & 25.9 & 58.9 & 23.1 & 30.4 & 3.2 & 37.5 & 14.8\\
\hspace{1em}2019 & 19.6 & 20.3 & 21.6 & 21.6 & 8.9 & 5.6 & 0.0 & 0.0\\
\addlinespace[0.3em]
\multicolumn{9}{l}{\textbf{Q4}}\\
\hspace{1em}2022 & 72.1 & 25.4 & 66.1 & 21.1 & 69.8 & 2.0 & 46.7 & 7.7\\
\hspace{1em}2021 & 71.9 & 25.8 & 73.2 & 23.2 & 75.0 & 4.3 & 43.5 & 7.7\\
\hspace{1em}2019 & 15.3 & 17.5 & 22.0 & 19.2 & 6.8 & 5.4 & 0.0 & 0.0\\
\bottomrule
\end{tabular}

        \end{center}
                \vspace{-5mm}
		\footnotesize{Each cell is the percent of markets originating from the specified airport with both carriers present in the market as either the ticketing carrier or operating carrier. Ticketing carriers are responsible for buying and selling of tickets while the operating carrier handles flight operations. Data for the first quarter of 2021 should be interpreted cautiously as this was before widespread vaccination availability.}
	\end{table}
    
	% As documented in Table \ref{tab:NEA_Exposure}, approximately 75\% of JetBlue's routes, passengers, and revenue and 15\% of American's routes, passengers, and revenue were in the markets directly impacted by the four cities within the NEA. As such, the vast majority of JetBlue's fleet was deployed in collaboration with American. The ruling against the agreement indicated concerns that this coordination lessened JetBlue's ability to compete in even markets that were not coordinated within, due to the limited ability of JetBlue to increase its fleet size.\footnote{As noted in the judgment, there is limited ability to fully understand the ramifications of the NEA on JetBlue's route restructuring over this time period as it is coterminous with the post-pandemic industry changes.}

    \begin{figure}
        \caption{Northeast Alliance Passenger Uptake}
        \label{fig:NEA_Uptake}
        \begin{center}
            \includegraphics[width = \linewidth]{05.Figures/NEA_OperationsGraph}
        \end{center}
        \vspace{-8mm}
        \footnotesize{A joint itinerary is one in which both JetBlue and American Airlines operated flights on one or more legs of the unidirectional trip. The ticketing carrier collects fares and issues tickets, the operating carrier operates the flights. Itineraries are classified as an "AA Ticket, JB Operator" if the entire itinerary was issued by American Airlines and JetBlue operated at least one leg of the trip.}
    \end{figure}

    Figure \ref{fig:NEA_Uptake} plots three measures of the implementation of the NEA: those of passengers on joint itineraries taking flights operated by both firms over the course of a unidirectional trip and those on itineraries ticketed to one carrier but operated by the other on at least one leg of the trip. This figure indicates that the vast majority of customers who used the NEA used it to book tickets with one carrier for flights operated, at least partially, by the other carrier. Interestingly, relatively few customers booked connecting fares with legs operated by different carriers. Furthermore, this plot is consistent with the notion that the NEA would not come into full effect until around the fourth quarter of 2021.\footnote{This reinforces the value of the replication exercises in this paper when compared to \citet{zou_assessing_2023} which only had access to data through the end of 2021 for its analysis.} 
    
	\subsection{Attempted JetBlue-Spirit Merger}
	\label{sec:Setting_Merger}
	In February 2022, Spirit announced its intention to merge with fellow ultra-low cost carrier Frontier \citep{schaper_frontier-spirit_2022}. This prompted a counteroffer from JetBlue in April for ownership of Spirit, which would lead to the attempted merger between Spirit and Frontier being called off in July amid a lack of support from Spirit shareholders \citep{josephs_jetblue_2022, josephs_spirit_2022}. By mid-October, Spirit shareholders approved the acquisition by JetBlue \citep{koenig_spirit_2022}. The next year would see the United States Department of Justice, the District of Columbia, Massachusetts, and New York file suit to block the merger in March \citep{chokshi_justice_2023}. Following a trial in the winter of 2023, the merger would be blocked on January 16, 2024, and the parties would ultimately decide against appealing the verdict \citep{chapman_jetblue_2024}.  These events are summarized in Table \ref{tab:JetBlue_Spirit_Timeline}. 

    	\begin{table}[tb]
		\caption{JetBlue-Spirit Merger Timeline}
		\label{tab:JetBlue_Spirit_Timeline}
		\begin{center}
			\begin{tabular}{ccc}
				\hline
				Year & Date & Event \\
				\hline
				2022 & February 7 & Frontier-Spirit Merger Announced \\
				& April 5 &  First JetBlue Offer for Spirit Released\\
				& May 6 & Spirit Rejects JetBlue Offer \\
				& July 27 &  Frontier-Spirit Merger Attempt Collapses\\
				& July 28 &  Spirit Board Approves JetBlue Merger\\
				& October 19 & Spirit Shareholders Approve Merger \\
				\hline
				2023 & March 7 &  Department of Justice Files Suit\\
				& October 31 - December 5 &  JetBlue-Spirit Merger Trial \\
				\hline
				2024 & January 16 & JetBlue-Spirit Merger Blocked \\
				& March 4 & JetBlue, Spirit Drop Appeal Plans \\
			\end{tabular}
		\end{center}
	\end{table}

	JetBlue publicly considered the acquisition of Spirit to be a top priority for the company, choosing to not appeal the ruling blocking its Northeast Alliance with American Airlines in favor of focusing its resources on overcoming the lawsuit seeking to block the merger with Spirit \citep{aratani_jetblue_2023}. Beyond these legal resources, it directed resources toward trying to win public favor over the merger. Notably, it coordinated comment submissions to a Department of Transportation regulatory filing regarding the merger with pro-merger comments sourced from its employees.\footnote{Some employees went on to dispute that these comments accurately reflected their views \citep{birnbaum_elizabeth_2023, birnbaum_jet-blue_2023}. In Appendix \ref{sec:NaturalLanguage}, I use stance detection techniques to analyze comments left on this filing in more detail.} Despite this, following the ruling against the merger it would ultimately choose to drop its appeals, with some financial analysts noting a significant deterioration of Spirit's financial stability between 2022 and 2024 \citep{sider_jetblue_2024}. 
	
	I now turn my attention to the ruling by Judge Young which ultimately blocked the JetBlue-Spirit merger attempt. The judgment identifies five key cities for his ruling: Orlando, San Juan, Miami and Fort Lauderdale (termed ``South Florida"), New York City, and Boston. These cities were identified on the basis that the majority of passengers in markets with competition between JetBlue and Spirit departed from these cities. These largely align with the cities in which the two firms would have the largest share of departing passengers within 2022 (Table \ref{tab:KeyCities}). However, the cities indicated ignores the firms' role in multiple smaller markets within Puerto Rico, namely Ponce and Aguadilla, where the two firms comprise the vast majority of the market. 
	
	In section 2.F of the judgment, the potential effects of the merger are listed as decreased airline seats, increased market concentration, increased debt for JetBlue, and increased prices for consumers. Table \ref{tab:JetBlueSpirit_Fleet} documents the aircraft in JetBlue and Spirit's fleets in 2022. Both airlines predominantly fly aircraft manufactured by Airbus. JetBlue's fleet is slightly more varied than Spirit's as it uses five different versions of the Airbus A321 aircraft\footnote{Two of these configurations reflect solely different seat configurations. The other three configurations are on the Airbus A321neo, a revision of the earlier aircraft.} Should Spirit's Airbus A320 and Airbus A321 have been adjusted to the predominant seat configurations of JetBlue's aircraft, a total of 20 seats would be lost on each Airbus A320 and 69 seats would be lost on each Airbus A321, for a total of 4,799 seats lost. This would reflect a loss of approximately 13\% of the seats on Spirit's aircraft.\footnote{If instead, Spirit's Airbus A321 were adjusted to the 200 seat configuration rather than the 159 configuration, then this would be 3500 seats lost or a loss of approximately 9.9\% of Spirit's seat capacity.} These rough calculations closely track the estimate of an 11\% reduction in seats accepted by the trial court in its decision.\footnote{The court further estimated a decline in annual departures of over 6.1 million. As I neither possess data on flight schedules nor model flight schedules endogenously, I am unable to assess this claim.} 
    
    \begin{table}
        \begin{center}
            \caption{JetBlue, Spirit Fleet Composition - 2022}
            \label{tab:JetBlueSpirit_Fleet}
            \vspace{-10mm}
          
\begin{tabular}[t]{llrrr}
\toprule
Manufacturer & Model & Seats & Count & Total Seats\\
\midrule
\addlinespace[0.3em]
\multicolumn{5}{l}{\textbf{JetBlue}}\\
\hspace{1em}Airbus & A220 & 140 & 14 & 1960\\
\hspace{1em}Airbus & A320 & 150 & 11 & 1650\\
\hspace{1em}Airbus & A320 & 162 & 119 & 19278\\
\hspace{1em}Airbus & A321 & 159 & 35 & 5565\\
\hspace{1em}Airbus & A321 & 200 & 28 & 5600\\
\hspace{1em}Airbus & A321neo & 138 & 5 & 690\\
\hspace{1em}Airbus & A321neo & 160 & 2 & 320\\
\hspace{1em}Airbus & A321neo & 200 & 16 & 3200\\
\hspace{1em}Embraer & E190 & 100 & 55 & 5500\\
\addlinespace[0.3em]
\multicolumn{5}{l}{\textbf{Spirit}}\\
\hspace{1em}Airbus & A319 & 145 & 31 & 4495\\
\hspace{1em}Airbus & A320 & 182 & 133 & 24206\\
\hspace{1em}Airbus & A321 & 228 & 30 & 6840\\
\bottomrule
\end{tabular}

        \end{center}
        \footnotesize{Source: B-43 Inventory Data.}
    \end{table}

    This paper estimates the counterfactual increase in market concentration and increase in prices in Section \ref{sec:Analysis_Merger}. 

	\section{Data and Summary Statistics}
	\label{sec:Data}
	The primary dataset used in the creation of this paper is the Bureau of Transportation Statistics' Airline Origin and Destination Survey (DB1B). The DB1B is a 10\% sample of all domestic airline itineraries within the United States. It includes data on pricing, distance, carrier, and number of connecting flights. Within the literature examining the aviation industry, it has been the preferred data for domestic air travel for decades (e.g. \citet{ciliberto_market_2021, berry_tracing_2010, goolsbee_how_2008, peters_evaluating_2006}). 

    Despite the breadth of its included information, the DB1B has one key limitation for this study - fares consist solely of the base airfare. As such, it does not include money spent on airline fees (such as for checked bag fees) nor on additional amenities. As Spirit and the other ultra-low cost carriers focus on ``unbundled" fares, these airline fares are liable to be reported as systematically lower than the full price paid by consumers after taking into account spending on auxiliary purchases.\footnote{Legacy carriers have instituted a tier of fare known within the industry as ``basic economy" to compete with Spirit through a limited amount of unbundled fares. As such, these fares are likewise underestimated. Unfortunately, the DB1B does not include reliable information on fare class, and as such, these fares are unable to be detected.} This inhibits a proper simulation of changes to consumer surplus following the merger. Consider a consumer who purchases an unbundled Spirit fare for \$30 along with a checked bag for \$50 but who had the option of purchasing a JetBlue ticket for \$90 which included a checked bag. Using the DB1B, I would estimate the consumer's change in surplus to be \$60 rather than \$10 following the merger due to an inability to assess full product characteristics. 
    
    However, despite this issue, it is still possible to gain insight into consumer welfare following the merger using this dataset. By analyzing the estimated change in the minimum fare within a market (in addition to the average fare), we can gain insight into how the merger would affect marginal entrants of the market who would have been liable to exit the market following the merger. This has the benefit of exhibiting similar logic to the judgment that blocked the merger, which focused on this group of travelers in justifying the decision to block the merger. 
    	
	Markets are defined by origin airport, destination airport, year, and quarter. Within this definition, originating and terminating airports are treated as the determinants of a market rather than airports' metropolitan statistical areas. This accounts for the known phenomenon that consumers do not treat airports within a metropolitan statistical area as interchangeable.\footnote{For an example of this phenomenon, \citet{goolsbee_how_2008} observes differential impacts on pricing of possible firm entry at the airport level than would be expected if airports within a metropolitan statistical area were treated as interchangeable by consumers.} Within this paper, products within a market are further defined by both ticketing carrier and non-stop status.\footnote{As such, each carrier can have either 0, 1, or 2 products within a market.} Appendix \ref{sec:DataProcessing} details the sample construction methodology and restrictions on markets and itineraries included within the sample (such as excluding markets that consist of airports that are fewer than 150 miles apart).
	
	Market size is defined as the geographic mean of the population of the origin and destination metropolitan statistical areas population. This is a standard assumption within this literature (such as in \citep{ciliberto_market_2021}). This is calculated using the United States Census Bureau's annual estimates of metropolitan statistical area population. The outside good within a market is defined as the decision not to consume air travel between an origin and destination airport pair. As such, the outside good includes not making a trip, making a trip by car or bus, and making a trip between two different airports within the same origin and destination metropolitan areas. 
    
	% For the analysis of the Northeast Alliance in Section \ref{sec:Analysis_NEA}, data on state level coronavirus cases from the Centers for Disease Control and Prevention was used. Airport markets were assigned to this data on the basis of the state of the principal city within the origin and destination metropolitan areas.\footnote{For example, the New York City market was assigned New York}. Coronavirus patients who were tested outside of their home county were included in the alternate county's testing data. As such, I believe that state level data is more reliable than county level data. Furthermore, the virus saw seasonal trends that were different between regions. This suggests that quarter level controls would be insufficient. Beyond this data, census bureau state-level and MSA-level estimates of personal income were used as controls for discretionary income.\footnote{There is a significant delay in releases of these estimates for MSA. As such, the state-level data allows for a longer sample to be analyzed. On the shared quarters, I do not observe any notable differences in the estimated coefficients through the use of state or MSA level income data.}

    \begin{table}
    \caption{Product Level Summary Statistics}
    \label{tab:Summary_Statistics_Product}
                    \vspace{-15mm}
                    \begin{center}
    
\begin{tabular}[t]{llllll}
\toprule
 & Mean & (SD) & Minimum & Median & Maximum\\
\midrule
\addlinespace[0.3em]
\multicolumn{6}{l}{\textbf{Pre-Pandemic}}\\
\hspace{1em}Price (2017 USD) & 234.01 & (68.85) & 33.12 & 236.38 & 810.58\\
\hspace{1em}Passengers & 4257.23 & (10192.88) & 100 & 810 & 192050\\
\hspace{1em}Distance (1000s) & 1.42 & (0.69) & 0.15 & 1.28 & 4.8\\
\hspace{1em}Extra Distance & 0.14 & (0.21) & 0 & 0.06 & 3.71\\
\hspace{1em}Nonstop & 0.28 & (0.45) & 0 & 0 & 1\\
\hspace{1em}Origin Destinations & 30 & (33.39) & 1 & 13 & 180\\
\hspace{1em}Origin Prescence (\%) & 36.26 & (31.28) & 0.54 & 19.57 & 100\\
\hspace{1em}Delta & 0.25 & (0.43) & 0 & 0 & 1\\
\hspace{1em}American & 0.22 & (0.41) & 0 & 0 & \vphantom{1} 1\\
\hspace{1em}United & 0.14 & (0.35) & 0 & 0 & 1\\
\hspace{1em}Southwest & 0.25 & (0.43) & 0 & 0 & 1\\
\hspace{1em}JetBlue & 0.03 & (0.17) & 0 & 0 & 1\\
\hspace{1em}Spirit & 0.03 & (0.18) & 0 & 0 & 1\\
\hspace{1em}Other Carrier & 0 & (0.06) & 0 & 0 & 1\\
\midrule
\hspace{1em}Observations & 307289 &  &  &  & \\
\midrule
\addlinespace[0.3em]
\multicolumn{6}{l}{\textbf{Post-Pandemic}}\\
\hspace{1em}Price (2017 USD) & 212.77 & (75.21) & 27.96 & 209.94 & 737.78\\
\hspace{1em}Passengers & 3531.43 & (8648.27) & 100 & 690 & 144930\\
\hspace{1em}Distance (1000s) & 1.41 & (0.67) & 0.15 & 1.28 & 3.86\\
\hspace{1em}Extra Distance & 0.14 & (0.19) & 0 & 0.07 & 1.83\\
\hspace{1em}Nonstop & 0.26 & (0.44) & 0 & 0 & 1\\
\hspace{1em}Origin Destinations & 29.24 & (33.72) & 1 & 12 & 187\\
\hspace{1em}Origin Prescence (\%) & 34.77 & (30.92) & 0.53 & 18.42 & 100\\
\hspace{1em}Delta & 0.22 & (0.41) & 0 & 0 & 1\\
\hspace{1em}American & 0.22 & (0.41) & 0 & 0 & 1\\
\hspace{1em}United & 0.13 & (0.34) & 0 & 0 & 1\\
\hspace{1em}Southwest & 0.26 & (0.44) & 0 & 0 & 1\\
\hspace{1em}JetBlue & 0.03 & (0.16) & 0 & 0 & 1\\
\hspace{1em}Spirit & 0.04 & (0.2) & 0 & 0 & 1\\
\hspace{1em}Other Carrier & 0.01 & (0.1) & 0 & 0 & 1\\
\midrule
\hspace{1em}Observations & 265196 &  &  &  & \\
\bottomrule
\end{tabular}

    \footnotesize{A product is defined as a set of origin airport, destination airport, year, quarter, firm, and nonstop status. ``Origin Destinations" is the number of airports served from the originating airport across all firms, ``Origin Prescence" is the fraction of these destinations served by the ticketing carrier. The pre-pandemic sample includes all quarters of the years 2017 through 2019. The post-pandemic sample includes data from the second quarter of 2021 through the second quarter of 2023.}
                    \end{center}
    \end{table}

    Summary statistics for product-level data are included in Table \ref{tab:Summary_Statistics_Product}. Both product prices and passengers fell on average following the pandemic, with the average itinerary becoming \$22 cheaper in real terms while having 700 fewer passengers. Notably, despite the post-pandemic period's economy having high levels of inflation, there is only a \$6 increase in nominal average fares between the two periods.  Delta slightly decreased as a share of total products by 3 percentage points while Southwest, Spirit, and the `Other' carrier increased their product offerings by approximately one percentage point. Products are slightly less likely to include an intermediate stop following the pandemic and cover slightly smaller distances. 

    Summary statistics for market-level characteristics are included in Table \ref{tab:Summary_Statistics_Market}. Despite the post-pandemic period having fewer overall markets included in the sample, JetBlue and Spirit competed in over a hundred additional markets in this period than in the three years before the pandemic. Additionally, there was on average roughly a third of an additional firm in the post-pandemic period in the average market, driven by firms operating a single product within the market. Finally, it is worth noting that the number of customers within the average market decreased by approximately 16,000 customers between the two periods. Aside from this, overall market characteristics are similar in terms of miles flown between the two periods. 

    \begin{table}
        \caption{Market Level Summary Statistics}
        \label{tab:Summary_Statistics_Market}
                \vspace{-15mm}
\begin{center}
            
\begin{tabular}[t]{llllll}
\toprule
 & Mean & (SD) & Minimum & Median & Maximum\\
\midrule
\addlinespace[0.3em]
\multicolumn{6}{l}{\textbf{Pre-Pandemic}}\\
\hspace{1em}Minimum Miles (1000s) & 1.18 & (0.64) & 0.15 & 1.02 & 2.95\\
\hspace{1em}Average Miles (1000s) & 1.23 & (0.66) & 0.15 & 1.07 & 4.39\\
\hspace{1em}Number of Firms & 2.93 & (1.49) & 1 & 3 & 9\\
\hspace{1em}Number of Products & 3.52 & (2.1) & 1 & 3 & 15\\
\hspace{1em}Number of Customers & 14974.29 & (28289.99) & 260 & 4150 & 406050\\
\hspace{1em}HHI & 8044.07 & (4347.52) & 1611.61 & 7058.6 & 56397.84\\
\midrule
\hspace{1em}Observations & 87363 &  & JetBlue Markets & 7442 & \\
\hspace{1em}JetBlue \& Spirit Markets & 1533 &  & Spirit Markets & 7474 & \\
\midrule
\addlinespace[0.3em]
\multicolumn{6}{l}{\textbf{Post-Pandemic}}\\
\hspace{1em}Minimum Miles (1000s) & 1.19 & (0.64) & 0.15 & 1.04 & 2.96\\
\hspace{1em}Average Miles (1000s) & 1.24 & (0.66) & 0.15 & 1.1 & 2.98\\
\hspace{1em}Number of Firms & 3.21 & (1.56) & 1 & 3 & 9\\
\hspace{1em}Number of Products & 3.79 & (2.16) & 1 & 3 & 14\\
\hspace{1em}Number of Customers & 13375.81 & (25085.61) & 230 & 3840 & 317370\\
\hspace{1em}HHI & 7479.76 & (4410.86) & 1460.46 & 6260.03 & 20000\\
\midrule
\hspace{1em}Observations & 70016 &  & JetBlue Markets & 5945 & \\
\hspace{1em}JetBlue \& Spirit Markets & 1554 &  & Spirit Markets & 9123 & \\
\bottomrule
\end{tabular}

            \footnotesize{A market is defined as a set of origin airport, destination airport, year, and quarter. The average miles reported within a market is weighted by itinerary passengers. The pre-pandemic sample includes all quarters of the years 2017 through 2019. The post-pandemic sample includes data from the second quarter of 2021 through the second quarter of 2023. JetBlue (Spirit) markets is the total number of markets operated by JetBlue (Spirit) within the period of observation. ``JetBlue \& Spirit Markets" is the total number of markets with both firms competing within them.}

\end{center}
    \end{table}

	\section{Analysis}
	\label{sec:Analysis}
	
	This section is organized into three subsections. I estimate a random-coefficient nested logit model of demand (Section \ref{sec:Analysis_Demand}), assume Bertrand-Nash competition with differentiated products to estimate marginal costs (Section \ref{sec:Analysis_Supply}), and then estimate three merger counterfactual simulations (a best case, an average case, and a worst case) (Section \ref{sec:Analysis_Supply}. 
	
	Before beginning these analyses, it is important to consider the implications of one critical assumption used within the simulation, namely that firms treat the overall market structure as exogenously determined when determining prices.\footnote{This is consistent with the internal pricing processes of airlines, discussed in \citet{hortacsu_organizational_2024}.} The most immediate consequence of this is that routing within the models discussed in this paper cannot respond to demand shocks within a given quarter. Furthermore, within the simulations, firms do not change which markets they operate in following the merger nor reposition their products within these markets except through price and changing the branding. % Two recent papers relax this assumption. In \citet{ciliberto_market_2021}, Meanwhile, \citet{li_repositioning_2022} estimates a static equilibrium model where firms can reposition their products' type (connecting or nonstop) in response to a merger. Their model predicts limited entry of new, nonstop products following three historical mergers. 
	
	This creates a problem for proper inference of the counterfactual world in which the merger had been completed. As discussed in the previous subsection, the NEA had included a reorganization of the route networks of each of the collaborating firms as part of the agreement.\footnote{One consequence of this is that NEA codeshare products ticketed to JetBlue are included as JetBlue products for purposes of the merger simulation.} As such, counterfactuals using markets between 2021 and 2023 are those in which the JetBlue-Spirit merger was completed while the NEA is in effect and without any resulting reorganization of routes. However, as the NEA would ultimately be ruled against before the trial for the Spirit merger, it is hard to believe in the existence of this world had the JetBlue-Spirit merger been allowed to be completed.
	
	Unfortunately, it is not possible to simply use markets from before the implementation of the NEA as this would require the usage of data from before the coronavirus pandemic. As documented in press sources and in a working paper \citep{ewen_zoom_2023}, air travel demand dynamics have greatly changed following the pandemic. In part thanks to the rise of telecommunications software such as Zoom, low-price elasticity business travel has lessened. Concurrently, American consumers acquired additional savings during the pandemic which they were able to use on additional consumption following the pandemic \citep{klitgaard_spending_2023}. As such, the change in the price elasticity of air travel following the pandemic is a priori ambiguous.  
	
 	To try to resolve these issues, I conduct all analyses within this section on two samples - the "pre-pandemic" sample from the first quarter of 2017 through the fourth quarter of  2019 and a "post-pandemic" sample consisting of data from the second quarter of 2021 through the end of the second quarter of 2023. By comparing the results from these two samples, a more complete picture can emerge of the counterfactual world in which the JetBlue-Spirit merger had been completed despite the issues facing each sample's overall credibility.  
 	 	 
 	 % Route Dynamics
 	 %	The Northeast Alliance between JetBlue and American poses an issue for proper counterfactual merger simulation between JetBlue and Spirit for two key reasons. The first of these is that the alliance saw the reorganization of JetBlue and American's route networks to allow them to compete more effectively against Delta and United within the New York and Boston markets. As my estimation of demand in Section \ref{sec:Analysis_Demand} treats network decisions as exogenous, this impedes proper counterfactual estimation.  The second is that through the codesharing agreement, each of these firms has products within my dataset that were not able to be fully offered by the firm alone.\footnote{This is further exacerbated by the restructuring of each firm's operations between these airports to have each route specialize on some routes.} In summation, there are observed products which would not have been offered absent the alliance while there are unobserved products which would have been offered absent the alliance. 
 	 
 	 %	Gaining an understanding of the breadth of the reorganization of the two airlines routes is difficult, but a helpful clue is present within the ruling against the agreement. The ruling notes that a goal of the reorganization in routes was to ensure that either JetBlue or American, but not both, were present in a given route between two cities. This suggests that understanding of the reorganization can be gained by estimating $P(AA \in R \mid JB \in R)$ and $P(JB \in R \mid AA \in R)$ as functions of a route being an "NEA Route" (that is, a route that touches on an airport in the New York City or Boston markets), distance and time.\footnote{Unlike in most other analysis of this paper, here we are interested in which airline is operating the planes on a given route, and so products are assigned to the operating carrier rather than the ticketing carrier.} These results are reported in Table \ref{tab:NEA_Op_Prob}.
 	 
 	 %	The observed products problem is easier to gain an understanding of. Figures \ref{fig:NEA_Mix_AAB6} and \ref{fig:NEA_Mix_B6AA} plot the number of estimated passengers on itineraries with American or JetBlue listed as the ticketed carrier and at least one leg of the journey on a flight operated by the other carrier. Two facts are apparent from this graph - the first is that despite the NEA agreement having been finalized in the first quarter of 2021, it took a good deal of time for JetBlue and American to allow their customers to book using the cross sharing provisions. The second is that from the first quarter of 2022 to the dissolution of the agreement in the third quarter of 2023, roughly 500,000 passengers would fly on itineraries made possible by the agreement. This accounts for roughly 5\% of the airlines' passengers for these NEA routes within this period. 
 	 	 
	\subsection{Demand Model and Results}
	\label{sec:Analysis_Demand}
	I use a random coefficient nested logit model to estimate demand, in line with the model originally documented in \citet{berry_automobile_1995}. Adopting the best notation described in \citet{conlon_best_2020}, each consumer $i$ in market $t$ has indirect utility from buying product $j$ as defined by 
	
	\[U_{ijt} = \delta_{jt} + \mu_{ijt} + \epsilon_{ijt}\]
	
	Where $\delta_{jt}$ is the mean utility across consumers in market $t$ for product $j$, $\mu_{ijt}$ is consumer level deviation from this mean utility, and $\epsilon_{ijt}$ is an unobserved consumer-level shock. $\delta_{jt}$ is parameterized as \[\delta_{jt} = \alpha p_{jt} + x_{j} \beta + F_{jt}\gamma  +  \epsilon_{jt}\] where $p_{jt}$ is the price of product $j$ in market $t$, $x_{jt}$ is a vector of observed itinerary characteristics\footnote{Namely nonstop status and miles flown}, and $\epsilon_{jt}$ is a product level shock shared by all consumers within a market. Within this model, $\mu_{ijt}$ is parameterized as \[\mu_{ijt} = \sigma_{p} p_{jt} \nu_{ip} + \sum_{k} \sigma_{k} x_{kjt} \nu_{ik} \] with the $\nu$ parameters drawn from a  normal distribution with mean zero and estimated variance. Finally, $\epsilon_{ijt}$ are assumed to arise from a type 1 extreme value distribution so that market shares will be of the discrete choice nested logit variety. Within this model specification, air fare is included within one nest while the outside good is included in the other nest.
	
	Consumer $i$ purchases itinerary $j$ if it has greater utility than all other products in the market. As such, market shares can be obtained by integrating over the consumers, resulting in the market share of each product being defined by \[s_{jt} = \int \frac{e^{\delta_{jt} + \mu_{ijt}}}{\sum_{j'} e^{\delta_{jt} + \mu_{ijt}}} d{\nu_{i}}\]
	
	Within this model, the contribution to utility shared by consumers, $x_{jt}$, contains the distance between the origin and destination airports within the market, the squared distance, a dummy variable which is one if an itinerary does not include any intermediate stops between the origin and destination airports, the difference between the miles flown by the itenary and the minimum number of miles flown within the market, the square of this difference, a dummy variable which is 1 if the origin or destination airport are in Florida or Las Vegas, and the ratio of the number of destinations served out of the origin airport by the carrier divided by the total number of destinations served out of that airport. This last measure is intended to proxy for rewards program strength of the carrier at the origin airport. $F_{jt}$ is a vector of controls which includes fixed effects for each year-quarter and carrier. The variables included in $x_{jt}$ are intended to be largely unresponsive to demand shocks - these characteristics of a product are determined primarily by a carrier's network structure and the geography between the origin and destination airports. As such, these characteristics should not change in response to unobserved quarterly demand shocks.
	
	 Four sets of instruments are used to account for the endogeneity of prices and shares within a market. The first set consists of a dummy variable which is 1 if at least one of the endpoint airports is a hub of the ticketing carrier, the product of this with the miles traveled on the itinerary, and the product of this with the square of the miles traveled. These serve as cost shifters.\footnote{In constructing the data, a round-trip fare is divided into two unidirectional fares each with imputed fare equal to half of the overall payment made. This motivates the symmetric treatment of both origin hubs and destination hubs in the construction of these instruments.} The second set of instruments, employed to account for endogeneity of market shares, consists of the differentiation instruments described in \citet{gandhi_measuring_2019} constructed from a dummy variable for nonstop flight status, the distance traveled, the square of the distance, and the service ratio of the ticketing carrier out of the destination airport. The third set of instruments, employed to instrument for the nesting parameter $\lambda$ consists solely of the number of products within a market to assist in model convergence. Finally, all remaining exogenous regressors and their interactions comprise the final set of instruments.\footnote{Other instruments for price were considered, including interactions between the gas miles variable and characteristics of the origin airport and interactions between the exogenous variables. However, the selection of price shifters used in the final model had the best performance across the tests documented in Tables \ref{tab:Instrument_Compare_Pre}. The final specification chosen (column 4) has the benefit of passing the Wu-Hausman test while failing the Test of Over Identification by the least amount of the tested models. As noted in \citet{nevo_measuring_2001}, provided enough observations it is virtually impossible to pass the over-identification test, and as such, I am not concerned with the result. For comparison purposes, the instrument comparison table on the main period of interest (2021 Q2 through 2023 Q2) is included as Table \ref{tab:PostPand_Instrument_Compare}.}
	
    Results for the estimation of this model's coefficients for both periods are included in Table \ref{tab:DemandEst}. With the exception of nonstop flight status and the tourist route dummy variable, all variables are predicted to influence consumer demand in both periods. However, of the variables which allow for random effects, only price takes a significant coefficient, of roughly $0.6$. Finally, I estimate a nesting parameter of a little over a tenth in both periods. This is consistent with high degrees of substitutability between air travel and the outside good, which is inconsistent with most previous literature's estimates of the nesting parameter. As such, consumers are predicted to have a high willing to enter (leave) the market in response to a price decrease (increase).

     JetBlue's products are more elastic than Spirit's, consistent with it targeting less budget conscious travelers than Spirit. Consumers are estimated to have become less price elastic between the pre-pandemic and post-pandemic periods. This is consistent with the notion that despite the decline in business travel following the pandemic, leisure travelers' spending patterns changed to be less price sensitive, perhaps due to excess savings acquired during the pandemic period or the desire to makeup for lost vacations.   
    
    \begin{table}
        \caption{Demand Estimation Results}
        \label{tab:DemandEst}
        \vspace{-15mm}
        \begin{center}
        
\begin{tabular}[t]{lll}
\toprule
Variable & Pre-Pandemic & Post-Pandemic\\
\midrule
\addlinespace[0.3em]
\multicolumn{3}{l}{\textbf{Linear Coefficients}}\\
\hspace{1em}Price & -3.053 & -3.114\\
\hspace{1em} & (0.38) & (0.44)\\
\hspace{1em}Nonstop & 0.838 & 1.163\\
\hspace{1em} & (1.3) & (0.8)\\
\hspace{1em}Miles Flown & 1.343 & 2.287\\
\hspace{1em} & (0.12) & (0.24)\\
\hspace{1em}Miles Flown$^2$ & -0.141 & -0.343\\
\hspace{1em} & (0.039) & (0.17)\\
\hspace{1em}Origin Prescence & 0.012 & 0.008\\
\hspace{1em} & (0.00045) & (4e-04)\\
\hspace{1em}Extra Miles & -2.653 & -2.337\\
\hspace{1em} & (0.11) & (0.13)\\
\hspace{1em}Extra Miles$^2$ & 1.617 & 0.527\\
\hspace{1em} & (0.069) & (0.11)\\
\hspace{1em}Tourist Route & 0.022 & 0.149\\
\hspace{1em} & (0.028) & (0.03)\\
\midrule
\addlinespace[0.3em]
\multicolumn{3}{l}{\textbf{Nonlinear Coefficients}}\\
\hspace{1em}Price & 0.591 & 0.599\\
\hspace{1em} & (0.14) & (0.12)\\
\hspace{1em}Nonstop & 0.249 & 0.081\\
\hspace{1em} & (5.4) & (10)\\
\hspace{1em}Miles Flown & 0.015 & 0.066\\
\hspace{1em} & (2.5) & (3.1)\\
\midrule
\addlinespace[0.3em]
\multicolumn{3}{l}{\textbf{Nesting Coefficient}}\\
\hspace{1em}Nesting Parameter & 0.132 & 0.115\\
\hspace{1em} & (0.047) & (0.032)\\
\midrule
\addlinespace[0.3em]
\multicolumn{3}{l}{\textbf{Summary Statistics}}\\
\hspace{1em}Period & 2017Q1-2019Q4 & 2021Q2-2023Q2\\
\hspace{1em}N Products & 307289 & 265196\\
\hspace{1em}N Markets & 87363 & 70016\\
\hspace{1em}Mean Elasticity & -5.591 & -5.211\\
\hspace{1em}Spirit Mean Elasticity & -4.28 & -3.44\\
\hspace{1em}JetBlue Mean Elasticity & -5.39 & -5.18\\
\hspace{1em}Mean Markup & 0.191 & 0.21\\
\midrule\\
\bottomrule
\end{tabular}

                \footnotesize{$^{***}p<0.01$; $^{**}p<0.05$; $^{*}p<0.1$ Products are defined as a Carrier-Nonstop pair within an Origin-Destination-Year-Quarter market. Origin Service Ratio is the fraction of direct routes out of the originating airport operated by the carrier divided by the number of distinct direct routes out fo that airport. Extra Miles is the average additional miles flown with a connecting itinerary minus the minimum miles flown within a market.  A tourist product is one that serves the Las Vegas metropolitan statistical area or an airport in Florida.}

        \end{center}
    \end{table}
		
	\subsection{Supply Model and Results}
	\label{sec:Analysis_Supply}
	The consumer aviation market is assumed to operate under Bertrand competition with differentiated products following the exogenous determination of quarterly route structure. This allows for recovery of marginal costs through the estimated demand elasticities.  These are included with the estimates of Demand within Table \ref{tab:DemandEst}. I estimate a slight increase in markups of approximately two percentage points between the pre- and post-pandemic periods. 
	
	
	\subsection{Spirit Merger Simulation}
	\label{sec:Analysis_Merger}
	I can now turn my attention to simulation of the JetBlue-Spirit merger. For each of the pre-pandemic and post-pandemic periods, I estimate three counterfactual mergers. These are respectively, a best case merger (where the merged product takes the lowest marginal cost and best unobservables of the two products), an average case merger (where the merged product takes the average of the two firms marginal costs and the average of the estimated unobservables), and a worst case scenario (where the merged product takes the greater of the marginal costs of the two firms and the lowest unobservable characteristics).\footnote{An additional simulation, not reported in this paper, solely reassigns the Spirit products to JetBlue, to capture short run pricing impacts of JetBlue internalizing Spirit's pricing decisions. However, simulation resulted in very little changes to prices and quantities. This is not altogether unsurprising, as the effects of Spirit on competition has generally been to promote the introduction of more varied fares \citep{shrago_spirit_2024}, which my data construction methodology abstracts from.} In each of these scenarios, I assume that the combined firm's connecting products take on the minimum of the miles flown, implicitly assuming that in all of these simulations that the combined firm will take advantage of better routing. %Finally, I estimate a non-merger counterfactual where Spirit exits every market. This final counterfactual is intended to represent the case in which Spirit goes bankrupt and is completely liquidated.  

    % Detail how this works mathematically
     
	 Table \ref{tab:Simulation_Price} contains the estimated price effects from the merger on individual product prices for markets in which both JetBlue and Spirit competed in. In both the best case scenario, I consistently estimate estimate declines in the average prices of products in markets wherein both JetBlue and Spirit competed of between 13\% and 15\%, or approximately \$23 and \$29. In the worst case scenario, I estimate a minimal increase in average market fares for the pre-pandemic period and an increase of approximately 5\% in the worst-case simulation. 
     
    \begin{table}
        \caption{Simulated Price Effects of Merger - Joint Markets}
        \label{tab:Simulation_Price}
                \vspace{-15mm}
        \begin{center}
         
\begin{tabular}[t]{lllllll}
\toprule
 & N & Mean & (SD) & Minimum & Median & Maximum\\
\midrule
\addlinespace[0.3em]
\multicolumn{7}{l}{\textbf{Pre-Pandemic}}\\
\addlinespace[0.3em]
\multicolumn{7}{l}{\textbf{Product Prices (100s, 2017 USD)}}\\
\hspace{1em}\hspace{1em}Observed & 12074 & 2.04 & (0.69) & 0.47 & 1.98 & 4.91\\
\hspace{1em}\hspace{1em}Best Case & 10106 & 2.08 & (0.66) & 0.46 & 2.02 & 5.07\\
\hspace{1em}\hspace{1em}Average Case & 10106 & 2.12 & (0.64) & 0.46 & 2.06 & 5.09\\
\hspace{1em}\hspace{1em}Worst Case & 10106 & 2.16 & (0.64) & 0.47 & 2.09 & 5.17\\
\addlinespace[0.3em]
\multicolumn{7}{l}{\textbf{Market Average Price (100s, 2017 USD)}}\\
\hspace{1em}\hspace{1em}Observed & 1418 & 2.01 & (0.43) & 0.93 & 1.95 & 3.1\\
\hspace{1em}\hspace{1em}Best Case & 1418 & 1.73 & (0.6) & 0.82 & 1.53 & 3.56\\
\hspace{1em}\hspace{1em}Average Case & 1418 & 2 & (0.51) & 1.01 & 1.92 & 3.4\\
\hspace{1em}\hspace{1em}Worst Case & 1418 & 2.01 & (0.5) & 0.98 & 1.91 & 3.39\\
\addlinespace[0.3em]
\multicolumn{7}{l}{\textbf{\% Change Average Price}}\\
\hspace{1em}\hspace{1em}Best Case & 1418 & -15.18 & (16.78) & -53.69 & -16.98 & 35.57\\
\hspace{1em}\hspace{1em}Average Case & 1418 & -0.76 & (10.49) & -41.21 & -0.64 & 31.1\\
\hspace{1em}\hspace{1em}Worst Case & 1418 & 0.19 & (10.26) & -37.48 & 0.4 & 45.58\\
\addlinespace[0.3em]
\multicolumn{7}{l}{\textbf{Post-Pandemic}}\\
\addlinespace[0.3em]
\multicolumn{7}{l}{\textbf{Product Prices  (100s, 2017 USD)}}\\
\hspace{1em}\hspace{1em}Observed & 13650 & 1.96 & (0.78) & 0.35 & 1.89 & 5.25\\
\hspace{1em}\hspace{1em}Best Case & 11496 & 2.01 & (0.77) & 0.4 & 1.94 & 5.33\\
\hspace{1em}\hspace{1em}Average Case & 11496 & 2.05 & (0.74) & 0.4 & 1.99 & 5.33\\
\hspace{1em}\hspace{1em}Worst Case & 11496 & 2.1 & (0.74) & 0.41 & 2.04 & 5.33\\
\midrule
\addlinespace[0.3em]
\multicolumn{7}{l}{\textbf{Market Average Price (100s, 2017 USD)}}\\
\hspace{1em}\hspace{1em}Observed & 1554 & 1.95 & (0.55) & 0.65 & 1.89 & 3.57\\
\hspace{1em}\hspace{1em}Best Case & 1554 & 1.72 & (0.68) & 0.61 & 1.69 & 3.67\\
\hspace{1em}\hspace{1em}Average Case & 1554 & 2.04 & (0.64) & 0.75 & 1.95 & 3.87\\
\hspace{1em}\hspace{1em}Worst Case & 1554 & 2.06 & (0.64) & 0.76 & 1.95 & 3.92\\
\addlinespace[0.3em]
\multicolumn{7}{l}{\textbf{\% Change Average Price}}\\
\hspace{1em}\hspace{1em}Best Case & 1554 & -13.58 & (18.18) & -58.98 & -11.03 & 38.23\\
\hspace{1em}\hspace{1em}Average Case & 1554 & 4.16 & (9.94) & -31.79 & 3.82 & 48.99\\
\hspace{1em}\hspace{1em}Worst Case & 1554 & 5.39 & (9.97) & -29.67 & 5.02 & 50.94\\
\bottomrule
\end{tabular}

         \footnotesize{Products from markets without both JetBlue and Spirit present are excluded. Distribution of changes of average market fares are graphed in Figures \ref{fig:AverageFare_ChangeDist_PrePandemic} and \ref{fig:AverageFare_ChangeDist_PostPandemic} for the pre-pandemic and post-pandemic periods respectively.}
        \end{center}
     \end{table}

    Notably, JetBlue was clear on its intention to retire the Spirit business model following the acquisition and retool all Spirit aircraft to follow the JetBlue configuration. As such, it is worth looking at the effect of this change on estimated pricing and costs on market fares in markets in which Spirit competed but not JetBlue. These estimated effects are notably smaller than the estimated effects of the merger on markets where Spirit operated but not JetBlue (Table \ref{tab:Simulation_Price_Spirit}).\footnote{As the different simulations are defined by the method used to combine products between the two firms, all simulations are equivalent for markets with only Spirit within them.} Within these markets, I estimate an average increase in average market fares of approximately 9.7\%, or approximately \$19. This is 

    \begin{table}
        \caption{Simulated Price Effects of Merger - Spirit Markets}
        \label{tab:Simulation_Price_Spirit}
                        \vspace{-15mm}
        \begin{center}
              
\begin{tabular}[t]{lllllll}
\toprule
 & N & Mean & (SD) & Minimum & Median & Maximum\\
\midrule
\addlinespace[0.3em]
\multicolumn{7}{l}{\textbf{Pre-Pandemic}}\\
\addlinespace[0.3em]
\multicolumn{7}{l}{\textbf{Prices (100s, 2017 USD)}}\\
\hspace{1em}\hspace{1em}Observed & 37776 & 2.1 & (0.78) & 0.38 & 2.08 & 7.17\\
\hspace{1em}\hspace{1em}Best Case & 30012 & 2.23 & (0.7) & 0.38 & 2.23 & 7.27\\
\hspace{1em}\hspace{1em}Average Case & 30012 & 2.23 & (0.7) & 0.38 & 2.23 & 7.26\\
\hspace{1em}\hspace{1em}Worst Case & 30012 & 2.23 & (0.7) & 0.38 & 2.23 & 7.26\\
\addlinespace[0.3em]
\multicolumn{7}{l}{\textbf{Market Average Price}}\\
\hspace{1em}\hspace{1em}Observed & 5735 & 2.08 & (0.46) & 0.7 & 2.08 & 4.61\\
\hspace{1em}\hspace{1em}Best Case & 5735 & 2.14 & (0.55) & 0.59 & 2.13 & 4.75\\
\hspace{1em}\hspace{1em}Average Case & 5735 & 2.14 & (0.54) & 0.57 & 2.13 & 4.75\\
\hspace{1em}\hspace{1em}Worst Case & 5735 & 2.14 & (0.55) & 0.59 & 2.13 & 4.75\\
\addlinespace[0.3em]
\multicolumn{7}{l}{\textbf{\% Change Average Price}}\\
\hspace{1em}\hspace{1em}Best Case & 5735 & 3.41 & (15.12) & -56.96 & 3.01 & 88.36\\
\hspace{1em}\hspace{1em}Average Case & 5735 & 3.27 & (15) & -58.43 & 2.84 & 88.35\\
\hspace{1em}\hspace{1em}Worst Case & 5735 & 3.38 & (15.11) & -55.75 & 2.95 & 88.35\\
\midrule
\addlinespace[0.3em]
\multicolumn{7}{l}{\textbf{Post-Pandemic}}\\
\addlinespace[0.3em]
\multicolumn{7}{l}{\textbf{Prices  (100s, 2017 USD)}}\\
\hspace{1em}\hspace{1em}Observed & 49630 & 1.97 & (0.82) & 0.29 & 1.95 & 6.27\\
\hspace{1em}\hspace{1em}Best Case & 40330 & 2.17 & (0.77) & 0.32 & 2.17 & 6.88\\
\hspace{1em}\hspace{1em}Average Case & 40330 & 2.17 & (0.77) & 0.32 & 2.17 & 6.75\\
\hspace{1em}\hspace{1em}Worst Case & 40330 & 2.17 & (0.77) & 0.32 & 2.17 & 6.81\\
\addlinespace[0.3em]
\multicolumn{7}{l}{\textbf{Market Average Price}}\\
\hspace{1em}\hspace{1em}Observed & 7569 & 1.97 & (0.53) & 0.6 & 1.94 & 4.07\\
\hspace{1em}\hspace{1em}Best Case & 7569 & 2.16 & (0.62) & 0.72 & 2.13 & 5.52\\
\hspace{1em}\hspace{1em}Average Case & 7569 & 2.16 & (0.62) & 0.72 & 2.12 & 5.52\\
\hspace{1em}\hspace{1em}Worst Case & 7569 & 2.16 & (0.63) & 0.73 & 2.13 & 5.52\\
\addlinespace[0.3em]
\multicolumn{7}{l}{\textbf{\% Change Average Price}}\\
\hspace{1em}\hspace{1em}Best Case & 7569 & 9.71 & (12.21) & -44.52 & 9.65 & 75.68\\
\hspace{1em}\hspace{1em}Average Case & 7569 & 9.7 & (12.23) & -44.26 & 9.57 & 75.68\\
\hspace{1em}\hspace{1em}Worst Case & 7569 & 9.71 & (12.23) & -44.66 & 9.7 & 75.68\\
\bottomrule
\end{tabular}

              \footnotesize{Products are solely those from markets in which Spirit operated but not JetBlue.}
        \end{center}
    \end{table}

     % Heterogeneity

     Now, I turn my attention to consumer welfare. Unfortunately, calculating consumer surplus is impossible to do with the available data. Spirit offers "unbundled" fares which have additional fees required for the various amenities included in the base price of competing firm's tickets (such as carry-on baggage). Consumers who would pay these fees at Spirit would have their change in consumer welfare overestimated by my merger simulation. Furthermore, as these fees differ between both customers and markets due to the use of algorithmic pricing \citep{senate_permanent_committee_on_investigations_majority_2024} it is infeasible to do back-of-the-envelope estimations to try to recover the ``true," fee inclusive, Spirit fares. 
     
     This motivates the consideration of another measure of consumer welfare, namely, the minimum fare available within each market. As noted by the judge in the merger trial, there exists consumers who would not be able to fly inside of a given market in the event of base airfare increasing. The overall change in this fare, in twenty dollar intervals, is expounded upon in Table \ref{tab:MinimumPrice}. Across all of my simulations, at least 14 markets are estimated to have the minimum price within the market increase by over \$80. Within both periods, these in the best case scenario markets are primarily between airports in the New York and Boston areas and various destinations within the Southern portion of the United States (such as Houston and Dallas). As such, it is consistent with the finding of the key markets of concern within the trial. 

    \begin{table}
        \caption{Change in Minimum Fare Available in Market}
        \label{tab:MinimumPrice}
                \vspace{-15mm}
        \begin{center}
            
\begin{tabular}[t]{lrrrrrr}
\toprule
\multicolumn{1}{c}{ } & \multicolumn{3}{c}{Pre-Pandemic} & \multicolumn{3}{c}{Post-Pandemic} \\
\cmidrule(l{3pt}r{3pt}){2-4} \cmidrule(l{3pt}r{3pt}){5-7}
 & Best & Average & Worst & Best & Average & Worst\\
\midrule
$<$ 20 & 1388 & 850 & 743 & 1426 & 848 & 763\\
20-40 & 73 & 453 & 306 & 56 & 390 & 254\\
40-60 & 35 & 168 & 278 & 36 & 213 & 256\\
60-80 & 22 & 44 & 168 & 25 & 77 & 180\\
80 $<$ & 15 & 18 & 38 & 11 & 26 & 101\\
\bottomrule
\end{tabular}

        \end{center}
        \footnotesize{The best case merger scenario is one in which the combined firm inherits the minimum average cost and greatest unobservables of each firm, the average case merger scenario has the combined JetBlue-Spirit inherit the average of the two firms' product characteristics, and the worst case scenario has the combined JetBlue-Spirit inherit the greatest marginal cost and lowest unobserveables. Prices are in 2017 dollars.}
    \end{table}    

    Now that the results on prices have been expounded upon, we can now turn our attention to how the simulations predict market concentration would change. Notably, as ultra-low cost airlines earn a significant amount of their income through axillary fares, which are not included within my dataset, estimation of market shares based on revenue are liable to understate these carriers within the market. As such, for these simulations, I calculate both an HHI calculated using market shares of airfare income as well as an HHI calculated using carrier shares of passengers within the market. These results are reported in Table  \ref{tab:HHI_Post}. Under every simulation model, I predict significant increases in market concentration under the passenger criterion, driven by an ascendant JetBlue. This is the opposite of the revenue shares criterion, which predicts significant declines in market concentration under the revenue criterion in all but the worst case scenarios. However, even under this more favorable metric, over 170 markets are still predicted to have increases about 100 points creating a structural presumption of illegality under the Department of Justice's 2023 merger guidelines. Overall, these results are consistent with a merger which would decrease consumer welfare in various markets while having effects that are pro-competitive at the national level.  

      \begin{table}
        \caption{Simulated Change in HHI}
        \label{tab:HHI_Post}
        \vspace{-15mm}
        \begin{center}
        
\begin{tabular}[t]{lrrrrrr}
\toprule
\multicolumn{1}{c}{ } & \multicolumn{3}{c}{Passenger Shares} & \multicolumn{3}{c}{Revenue Shares} \\
\cmidrule(l{3pt}r{3pt}){2-4} \cmidrule(l{3pt}r{3pt}){5-7}
 & Best & Average & Worst & Best & Average & Worst\\
\midrule
\addlinespace[0.3em]
\multicolumn{7}{l}{\textbf{Pre-Pandemic}}\\
\hspace{1em}$<$ 0 & 294 & 276 & 392 & 795 & 518 & 446\\
\hspace{1em}0 - 100 & 42 & 63 & 46 & 63 & 58 & 42\\
\hspace{1em}100 - 1000 & 347 & 495 & 481 & 414 & 553 & 491\\
\hspace{1em}1000 - 3000 & 429 & 583 & 516 & 248 & 383 & 478\\
\hspace{1em}3000 $<$ & 421 & 116 & 98 & 13 & 21 & 76\\
\midrule
\addlinespace[0.3em]
\multicolumn{7}{l}{\textbf{Post-Pandemic}}\\
\hspace{1em}$<$ 0 & 549 & 441 & 287 & 1202 & 808 & 346\\
\hspace{1em}0 - 100 & 52 & 133 & 115 & 57 & 157 & 117\\
\hspace{1em}100 - 1000 & 303 & 585 & 748 & 235 & 472 & 761\\
\hspace{1em}1000 - 3000 & 267 & 323 & 386 & 60 & 116 & 324\\
\hspace{1em}3000 $<$ & 383 & 72 & 18 & 0 & 1 & 6\\
\bottomrule
\end{tabular}

        \end{center}
    \end{table}

    
    \begin{figure}
    \caption{Simulated Change in Pre-Pandemic Minimum Market Fares}
    \includegraphics[width = \linewidth]{PrePandemic_Merger_Change_MinimumFare_Dist}
    \label{fig:PrePan_MinimumFare_Dist}
    The mean change in markets' minimum fares is 7.25 (19.36) [26.15] in the best (average) [worst] case merger simulations respectively.  
    \end{figure}    

    \begin{figure}
    \caption{Simulated Change in Post-Pandemic Minimum Market Fares}
    \includegraphics[width = \linewidth]{Merger_Change_MinimumFare_Dist}
    \label{fig:PostPan_MinimumFare_Dist}
    The mean change in markets' minimum fares is 6.51 (21.45) [29.21] in the best (average) [worst] case merger simulations respectively.  
    \end{figure}    

    Figures \ref{fig:PrePan_MinimumFare_Dist} and \ref{fig:PostPan_MinimumFare_Dist} graph the distribution of estimated minimum market fares for the pre-pandemic and post-pandemic periods. In the pre-pandemic (post-pandemic) period there exists roughly 300 (200) markets estimated to have decreases of under \$5 in the minimum market fare in all specifications. These markets are characterized by ... Beyond this, it is clear that under the average case and worst case scenarios, it is clear that rather than a simple uniform increase in price caused by the merger, the distribution changes to be higher variance with a higher mean.  

  %  \subsubsection{Spirit Exit Simulation}
  % Following the failure of the merger with JetBlue to consummate, Spirit Airlines would file for chapter 11 bankruptcy protection. Within this section, I estimate the counterfactual pricing effects of Spirit exiting every market, which would occur should it ultimately need to change its filing to be for chapter 7 bankruptcy protection. 
    
	\section{Conclusion}
	\label{sec:Conclusion}

    Within this paper, I retrospectively analyzed the Northeast Alliance's effect on fares and codesharing tickets while prospectively simulating the thwarted JetBlue-Spirit merger's impacts on airfare. I find mixed evidence for anti-competitive effects arising from the Northeast Alliance. In contrast, I find counterfactual price effects from the JetBlue-Spirit merger consistent with the ruling in the case. I estimate that this merger would have harmed extremely price sensitive travelers by raising airfares by over \$60 dollars in over 35 markets with both the pre-pandemic and post-pandemic periods. This effect persists between different simulations, with the worst case scenario predicting over 200 markets having price increases over \$60 in both periods. 
	
	\pagebreak 
	\bibliography{airline} 
	\bibliographystyle{abbrvnat.bst}
	\FloatBarrier
	
\pagebreak 
\begin{appendices}
	
	\section{Data Processing Methodology}
	\label{sec:DataProcessing}
	As detailed in Section \ref{sec:Data},	the Bureau of Transportation Statistics' Airline Origin and Destination Survey (DB1B) database was the primary data set used for this research. After compiling the DB1B into a single dataset for the years 2015 through the second quarter of 2023, some observations were excluded from the sample. Itineraries with fares lower than \$15 were excluded to remove air travel purchased through frequent flier rewards points ({X}\% of fares excluded). Similarly, in line with prior work\footnote{such as \citet{berry_tracing_2010}}, fares of over \$1,500 dollars were excluded to avoid fares which were erroneously recorded (X\% of fares were excluded this way). Beyond fares, itineraries were excluded from the sample if they had three or more layovers\footnote{These reflect substantially distinct consumption behavior. A total of X\% of observations were excluded this way.} or if they had a leg outside of the continental United States.\footnote{As noted in \citet{ciliberto_market_2021}, these flights receive subsidies from the United States Postal Service. As such, proper marginal cost recovery is infeasible while including them in the sample.} 
	
	Beyond this excluding of individual itineraries, additional filtering rules were placed on products and markets. All markets within the year 2020 and the first quarter of 2021 were dropped to avoid contamination from the worst of the coronavirus pandemic. Furthermore, markets were excluded if they had fewer than 500 passengers fly within them, or had origin and destination airports closer than 150 miles. This restriction is in line with the past-literature (such as \citet{ciliberto_does_2014}) and serves to not only improve computational speed but also account for these markets facing stronger competition from the outside good than other, more distant markets. Finally, products with fewer than 100 passengers were excluded from the sample to reduce the influence of product offerings that are liable to be nonstandard within the market and thus do not meaningfully contribute to competition. 
    	
	In calculating product shares, the total number of buyers of each product is considered to be ten times the number of passengers who bought it as the DB1B is a 10\% sample of the data. Following this, the hundred largest airports by passenger flows in the second quarter of 2022 were identified\footnote{This was chosen as the last quarter before the board of Spirit approved the merger.} and all other airports were excluded (X\% percent of markets representing X\% of consumers were excluded this way). This was done to improve computational speed. Notably, as both firms of interest within this paper generally compete in larger airport markets, this has minimal impact on the results.  
	
	% Beyond the handling of data acquired from the DB1B, data on the daily spot price of jet fuel was acquired through the U.S. Energy Information Administration. This was averaged at the quarterly level. 
	
	As part of the handling of price data, prices were modified in two ways. For Spirit itineraries completed before 2020, fares had an additional \$22.99 times the number of trip legs added to them. This accounts for Spirit's additional usage fee placed on itineraries which were not booked in-person at the airport, and which the majority of consumers paid.\footnote{As documented in \citet{shrago_spirit_2024}, these fees were included in DB1B releases following 2020.} Following this modification, prices were re-expressed in terms of 2017 United States dollars to account for inflation.

    \FloatBarrier
	\section{Merger Comments Analysis}
	\label{sec:NaturalLanguage}

As part of the merger process, JetBlue and Spirit were required to file an application with the Department of Transportation for the transference of operating certificates from Spirit to the combined firm, to be effective after the completion of the merger. Members of the public were allowed to leave public comments on the regulatory filing. Within this section, I employ stance detection techniques to analyze these comments at scale. While these comments are largely irrelevant to the result of this particular merger (namely, that it would be rejected following a suit brought by the Department of Justice), the use of these comments was used as part of the two firms' attempt to get the public to approve of the merger.

Stance detection, in brief, is the task of detecting the position held by the author of a text regarding some topic. In this context, it is to determine if the author of a comment left on the regulatory filing supported or opposed the proposed JetBlue-Spirit merger. This context is particularly suitable for the use of modern machine learning models as focused and direct. Therefore, an unsupervised zero-shot model should be effective with minimal issues with trying to gauge potentially contradictory statements that could be found in a longer work. 

The stance detection problem should not be confused with that of the sentiment analysis problem. Sentiment analysis intends to capture the emotions expressed in a text rather than identify the emotions expressed within the text. As an example of how these differ, consider the comment ``Competition is good for a healthy economy."\footnote{This is an actual comment left on the regulatory docket.} Using a pre-trained sentiment detection model developed for analyzing financial sentiment data, FinBERT, \footnote{Model documentation is contained in \citet{araci_finbert_2019}.} this statement is judged to possess positive sentiment. However, it is correctly judged to oppose the merger by the stance detection model used for my analysis. Table \ref{tab:Sentiment_Stance_Table} details the breakdown of unique comments' assigned sentiments and stances. 

\begin{table}
    \caption{Sentiment and Stance - Unique Comments}
    \label{tab:Sentiment_Stance_Table}
    \vspace{-15mm}
    \begin{center}
    
\begin{tabular}[t]{llll}
\toprule
\multicolumn{1}{c}{Stance} & \multicolumn{3}{c}{Sentiment} \\
\cmidrule(l{3pt}r{3pt}){1-1} \cmidrule(l{3pt}r{3pt}){2-4}
 & Positive & Neutral & Negative\\
\midrule
Approves & 40 & 37 & 4\\
Disapproves & 5 & 387 & 227\\
\bottomrule
\end{tabular}

    \end{center}
        \vspace{-5mm}
    \footnotesize{Each cell includes the number of comments with the given stance and sentiment. Only unique comments are included within this table. }
\end{table}

This paper uses the pre-trained model documented in \citet{laurer_less_2024} to detect the stances of each comment left on the docket. Each comment is assessed for the probability that each comment agrees with the statements ``The author of this comment \{approves of, disagrees with\} the merger." As these statements are mutually exclusive, the probabilities assigned for each comment sum to 1. As documented in Figures \ref{fig:ProbabilityApprove} and \ref{fig:ProbabilityApprove_Unique}, most comments are strongly polarized, suggesting that the language model had little difficulty in assigning stances to comments. Looking over a sample of fifty unique comments, all are sorted as would be expected based on my understanding of the text. As such, I believe that this model is well suited for analyzing the public comments.  

	\begin{figure}
		\caption{Probability Comments Approve}
		\label{fig:ProbabilityApprove}
        \begin{center}
        \includegraphics{05.Figures/stance_strength_graph}
        \end{center}
		\begin{minipage}{\textwidth} 
			{\footnotesize Data is sourced from the Department of Transportation regulatory filing regarding the JetBlue-Spirit merger (DOT-OST-2023-0024). ``Probability Approve" is the probability that a comment approves of the merger.} 
		\end{minipage}
	\end{figure}
	
	\begin{figure}
		\caption{Probability Comment Approves - Unique Comments Only}
		\label{fig:ProbabilityApprove_Unique}
        \begin{center}
            \includegraphics{05.Figures/stance_strength_unique.pdf}
        \end{center}
				\begin{minipage}{\textwidth} 
			{\footnotesize Data is sourced from the Department of Transportation regulatory filing regarding the JetBlue-Spirit merger (DOT-OST-2023-0024). ``Probability Approve" is the probability that a comment approves of the merger.} 
		\end{minipage}
	\end{figure}


Table \ref{tab:Stance_Summary} contains summary statistics for these comments. Most comments approve of the merger. However, this is driven by duplicate comments. \footnote{The exact legitimacy of these duplicate comments was a matter of some public debate, with some lawmakers alleging that they represented an "astroturf campaign" by the two merging firms falsely attributing them to employees \citep{birnbaum_elizabeth_2023}.}  The vast majority of unique comments, on the other hand, disapproved of the merger. On average, comments which approve of the merger are longer than those that disapprove. This table provides a helpful demonstration of the difference between the stance detection and sentiment detection problems -  the majority of disapproving comments expressed their views with neutral sentiment. Finally, the table documents the state of origin for the comments. 

\begin{table}[h]
    \caption{Stance Detection Summary Statistics}
    \label{tab:Stance_Summary}
    
\begin{tabular}[t]{llllll}
\toprule
 & Mean & (SD) & Minimum & Median & Maximum\\
\midrule
P(Approves) & 0.89 & (0.31) & 0 & 1 & 1\\
Approving Comment P(Approves) & 1 & (0.02) & 0.51 & 1 & 1\\
Disapproving Comment P(Approves) & 0.01 & (0.04) & 0 & 0 & 0.49\\
New York Comment & 0.14 & (0.35) & 0 & 0 & 1\\
Florida Comment & 0.35 & (0.48) & 0 & 0 & 1\\
\addlinespace
Massachusetts Comment & 0.05 & (0.22) & 0 & 0 & 1\\
Puerto Rico Comment & 0.01 & (0.12) & 0 & 0 & 1\\
\midrule
Observations & 10185 &  &  &  & \\
\bottomrule
\end{tabular}

    \begin{minipage}{\textwidth} 
        {\footnotesize Data is sourced from the Department of Transportation regulatory filing regarding the JetBlue-Spirit merger  (DOT-OST-2023-0024). Comments have the stance with the highest probability assigned to them. This is the ``Stance Probability." Similarly, ``Sentiment Assigned Probability" is the sentiment with the highest  probability assigned to a comment by the language model. Comment length is in characters.} 
    \end{minipage}
\end{table}

Figure \ref{fig:CommentTimeline} plots the distribution of submitted comments on each day after the regulation was available for commenting upon. In the first twenty days, virtually every comment left on the docket supported the merger. Virtually every comment left on the docket after this period was opposed the merger. This may reflect asymmetry in the the resources available to JetBlue, Spirit, and anti-merger consumer welfare organizations.  

    \begin{figure}[h]
		\caption{Timeline of Submitted Comments}
		\label{fig:CommentTimeline}
        \begin{center}
    	\includegraphics{stance_submission_timeline}
        \end{center}
		\begin{minipage}{\textwidth} 
			{\footnotesize Data is sourced from the Department of Transportation regulatory filing regarding the JetBlue-Spirit merger  (DOT-OST-2023-0024). Comments have the stance with the highest probability assigned to them.} 
		\end{minipage}
	\end{figure}
	
	
	

	\FloatBarrier
	\pagebreak
	\section{Additional Figures and Tables}	

	\subsection{Additional Descriptive Figures and Tables: Northeast Alliance}
	
%	\begin{landscape}
%		\begin{table}
%			\caption{Exposure to Northeast Alliance}
%			\label{tab:NEA_Exposure}
%			\include{06.Tables/NEA_Percentage_Tied_Up}
%			\begin{minipage}{\textwidth} % choose width suitably
%				{\footnotesize A "NEA Route" is a direct flight with either origin or destination that is BOS, JFK, LGA, EWR. "\% NEA Passengers" refers to the share of an airline's passengers that had itinerary with origin, destination, or intermediate stop at one of BOS, JFK, LGA, EWR. "\% NEA Revenues"  is the percentage of an airline's revenue within the domestic United States generated by the aforementioned passengers, calculated by multiplying the number of passengers of an itinerary by its average fare. It does not account for inter firm transfers in the case of JetBlue and American.}
%			\end{minipage}
%		\end{table}
%	\end{landscape}

\begin{landscape}
    \begin{table}
        \caption{NEA Codesharing Products}
        \label{tab:NEA_Codesharing_Table}
        \vspace{-15mm}
        \begin{center}
        
\begin{tabular}[t]{lrrrrrrrrrrr}
\toprule
 & 2021Q1 & 2021Q2 & 2021Q3 & 2021Q4 & 2022Q1 & 2022Q2 & 2022Q3 & 2022Q4 & 2023Q1 & 2023Q2 & 2023Q3\\
\midrule
\addlinespace[0.3em]
\multicolumn{12}{l}{\textbf{All Markets}}\\
\hspace{1em}JetBlue Only & 843 & 1375 & 1327 & 1374 & 1363 & 1583 & 1531 & 1456 & 1412 & 1562 & 1640\\
\hspace{1em}JetBlue American Required & 10 & 22 & 25 & 17 & 24 & 27 & 21 & 29 & 18 & 41 & 35\\
\hspace{1em}JetBlue-American Joint & 0 & 0 & 0 & 67 & 90 & 69 & 50 & 85 & 74 & 134 & 90\\
\hspace{1em}American Only & 4487 & 6728 & 7534 & 7351 & 7119 & 7596 & 8099 & 8338 & 7742 & 8173 & 8434\\
\hspace{1em}American JetBlue Required & 61 & 90 & 86 & 87 & 84 & 88 & 87 & 75 & 81 & 81 & 86\\
\hspace{1em}American-JetBlue Joint & 1 & 0 & 0 & 1 & 0 & 0 & 2 & 1 & 2 & 6 & 3\\
\addlinespace[0.3em]
\multicolumn{12}{l}{\textbf{Spirit Markets}}\\
\hspace{1em}JetBlue Only & 338 & 438 & 420 & 447 & 409 & 517 & 490 & 540 & 554 & 644 & 676\\
\hspace{1em}JetBlue American Required & 4 & 11 & 6 & 9 & 10 & 15 & 12 & 14 & 11 & 17 & 12\\
\hspace{1em}JetBlue-American Joint & 0 & 0 & 0 & 22 & 26 & 20 & 16 & 27 & 25 & 44 & 33\\
\hspace{1em}American Only & 1161 & 1491 & 1694 & 1849 & 1773 & 2012 & 2112 & 2340 & 2309 & 2563 & 2666\\
\hspace{1em}American JetBlue Required & 16 & 28 & 28 & 30 & 27 & 31 & 16 & 23 & 21 & 25 & 28\\
\hspace{1em}American-JetBlue Joint & 0 & 0 & 0 & 0 & 0 & 0 & 0 & 0 & 0 & 0 & 1\\
\bottomrule
\end{tabular}

        \end{center}
        \footnotesize{Each row indicates the number of products within a given quarter that belong to each classification. A "Firm Only" product is one in which the product had no codesharing flights included within it. A "JetBlue American Required" product is one in which all itineraries credited to the JetBlue for that quarter had all legs operated by American. A "JetBlue-American Joint" product is one where at least one leg was operated by each firm. A Spirit Market is one in which Spirit operated within it during the given quarter of interest. }
    \end{table}
\end{landscape}
    \begin{figure}[h]
        \caption{NEA: Overlapped Operating Routes}
        \label{fig:NEA_Operating}
        \includegraphics[width = \linewidth]{05.Figures/NEA_Operating_Graph.pdf}
        \footnotesize{Figure plots the number of routes operated by both JetBlue and American within a given quarter. The vertical line represents the start of the Northeast Alliance in January 2021.}
    \end{figure}

    \FloatBarrier\pagebreak

	\subsection{Descriptive Figures and Tables: JetBlue, Spirit}
	\begin{figure}[h]
	\caption{JetBlue, Spirit Fleet Size Over Time}
	\label{fig:Both_fleet}
	\includegraphics[width = \linewidth]{Both_Planes.pdf}
	\footnotesize{Source: B-43 Inventory Data. Each bar is the number of airplanes in a given firm's inventory within a given year.}
\end{figure}

\begin{figure}
	\caption{JetBlue, Spirit Airports - 2022}
	\label{fig:JBSpirit_Airports_2022}
	\includegraphics[width = \linewidth]{Map_Mainland_Both_2022.pdf}
	\footnotesize{Derived from DB1B Data. Beyond the United States mainland, both carriers operated in Puerto Rico.}
\end{figure}


	\begin{table}
		\caption{JetBlue and Spirit: Overlap Cities - 2022}
		\label{tab:KeyCities}
		
\begin{tabular}{lrrr}
\toprule
City & Firm Passengers & Total Passengers & Share\\
\midrule
Ponce, PR & 106320 & 106320 & 1.000\\
Aguadilla, PR & 251180 & 321170 & 0.782\\
San Juan, PR & 1848180 & 4149260 & 0.445\\
Boston, MA & 4262240 & 12136460 & 0.351\\
West Palm Beach/Palm Beach, FL & 919690 & 2960650 & 0.311\\
\addlinespace
Miami, FL & 5885260 & 19049140 & 0.309\\
Charlotte Amalie, VI & 155220 & 584450 & 0.266\\
New York, NY & 8243150 & 32401400 & 0.254\\
Hartford, CT & 596840 & 2358950 & 0.253\\
Orlando, FL & 4890200 & 19981730 & 0.245\\
\addlinespace
Fort Myers, FL & 964970 & 4577540 & 0.211\\
Detroit, MI & 1330090 & 7481070 & 0.178\\
Cleveland, OH & 567000 & 3537960 & 0.160\\
Richmond, VA & 235760 & 1474130 & 0.160\\
New Orleans, LA & 774190 & 4909390 & 0.158\\
\addlinespace
Las Vegas, NV & 2783710 & 18384770 & 0.151\\
Tampa, FL & 1371860 & 9955070 & 0.138\\
Pittsburgh, PA & 391900 & 3023570 & 0.130\\
Los Angeles, CA & 2839960 & 22400620 & 0.127\\
Philadelphia, PA & 844170 & 7694760 & 0.110\\
\bottomrule
\end{tabular}

		\footnotesize{Derived from DB1B Data. Cities are ordered by the combined share of passengers who used JetBlue or Spirit flights as a share of the total passengers departing from the city within 2022. Cities in which only one firm operates are excluded.}
	\end{table}

	
	\FloatBarrier
	
\FloatBarrier
    \subsection{Additional Merger Figures}
    \begin{table}
        \caption{Market Level Summary Statistics (By Competition Structure)}
        \label{fig:Market_Type_Summary}
                  \vspace{-15mm}
        \begin{center}
        
\begin{tabular}[t]{llllll}
\toprule
 & Mean & (SD) & Minimum & Median & Maximum\\
\midrule
\addlinespace[0.3em]
\multicolumn{6}{l}{\textbf{Pre-Pandemic}}\\
\addlinespace[0.3em]
\multicolumn{6}{l}{\textbf{Spirit Markets}}\\
\hspace{1em}\hspace{1em}Minimum Miles (1000s) & 1.3 & (0.61) & 0.18 & 1.14 & 2.81\\
\hspace{1em}\hspace{1em}Average Miles (1000s) & 1.35 & (0.66) & 0.18 & 1.19 & 4.39\\
\hspace{1em}\hspace{1em}Number of Firms & 4.8 & (1.37) & 1 & 5 & 9\\
\hspace{1em}\hspace{1em}Number of Products & 6.53 & (2.25) & 1 & 7 & 14\\
\hspace{1em}\hspace{1em}Number of Customers & 51859.49 & (48793.03) & 300 & 37600 & 360320\\
\hspace{1em}\hspace{1em}HHI & 8225.48 & (4382.4) & 1622.69 & 7360.39 & 56397.84\\
\midrule
\hspace{1em}\hspace{1em}Observations & 5941 &  &  &  & \\
\midrule
\addlinespace[0.3em]
\multicolumn{6}{l}{\textbf{JetBlue \& Spirit Markets}}\\
\hspace{1em}\hspace{1em}Minimum Miles (1000s) & 1.44 & (0.71) & 0.32 & 1.17 & 2.78\\
\hspace{1em}\hspace{1em}Average Miles (1000s) & 1.47 & (0.73) & 0.32 & 1.19 & 2.9\\
\hspace{1em}\hspace{1em}Number of Firms & 6.17 & (1.06) & 2 & 6 & 9\\
\hspace{1em}\hspace{1em}Number of Products & 8.51 & (2) & 3 & 8 & 15\\
\hspace{1em}\hspace{1em}Number of Customers & 80495.06 & (53421.42) & 1300 & 69680 & 344530\\
\hspace{1em}\hspace{1em}HHI & 6750.6 & (2646.51) & 1680.29 & 6247.49 & 17066.21\\
\midrule
\hspace{1em}\hspace{1em}Observations & 1533 &  &  &  & \\
\midrule
\addlinespace[0.3em]
\multicolumn{6}{l}{\textbf{Post-Pandemic}}\\
\addlinespace[0.3em]
\multicolumn{6}{l}{\textbf{Spirit Markets}}\\
\hspace{1em}\hspace{1em}Minimum Miles (1000s) & 1.32 & (0.6) & 0.18 & 1.22 & 2.81\\
\hspace{1em}\hspace{1em}Average Miles (1000s) & 1.37 & (0.63) & 0.18 & 1.26 & 2.92\\
\hspace{1em}\hspace{1em}Number of Firms & 5.11 & (1.24) & 2 & 5 & 8\\
\hspace{1em}\hspace{1em}Number of Products & 6.55 & (2.01) & 2 & 6 & 13\\
\hspace{1em}\hspace{1em}Number of Customers & 36898.62 & (37432.1) & 410 & 24260 & 214550\\
\hspace{1em}\hspace{1em}HHI & 7648.74 & (3855.01) & 1750.61 & 7023.19 & 19917.19\\
\midrule
\hspace{1em}\hspace{1em}Observations & 7569 &  &  &  & \\
\midrule
\addlinespace[0.3em]
\multicolumn{6}{l}{\textbf{JetBlue \& Spirit Markets}}\\
\hspace{1em}\hspace{1em}Minimum Miles (1000s) & 1.52 & (0.67) & 0.2 & 1.26 & 2.79\\
\hspace{1em}\hspace{1em}Average Miles (1000s) & 1.56 & (0.69) & 0.2 & 1.35 & 2.9\\
\hspace{1em}\hspace{1em}Number of Firms & 6.34 & (0.94) & 3 & 6 & 9\\
\hspace{1em}\hspace{1em}Number of Products & 8.78 & (1.78) & 3 & 9 & 14\\
\hspace{1em}\hspace{1em}Number of Customers & 73002.09 & (54873.21) & 2250 & 59070 & 261740\\
\hspace{1em}\hspace{1em}HHI & 6635.64 & (2769.53) & 1730.25 & 6207.7 & 17230.6\\
\midrule
\hspace{1em}\hspace{1em}Observations & 1554 &  &  &  & \\
\midrule
\bottomrule
\end{tabular}

        \end{center}
    \end{table}
    
    \begin{table}
        \caption{Pre-Pandemic Instrument Comparison Table}
        \label{tab:Instrument_Compare_Pre}
        \resizebox{\linewidth}{!}{%

\begin{tabular}{l c c c c c c c c c}
\toprule
 & Model 1 & Model 2 & Model 3 & Model 4 & Model 5 & Model 6 & Model 7 & Model 8 & Model 9 \\
\midrule
Price                       & $-0.42^{***}$ & $-4.97^{***}$ & $-0.18^{**}$  & $-0.18^{**}$  & $-2.12^{***}$ & $-0.14^{*}$   & $-2.09^{***}$ & $-2.28^{***}$ & $-2.23^{***}$ \\
                            & $(0.00)$      & $(0.09)$      & $(0.06)$      & $(0.06)$      & $(0.04)$      & $(0.06)$      & $(0.04)$      & $(0.03)$      & $(0.03)$      \\
Nesting                     & $0.55^{***}$  & $0.42^{***}$  & $-0.14^{***}$ & $-0.14^{***}$ & $0.12^{***}$  & $-0.14^{***}$ & $0.12^{***}$  & $0.18^{***}$  & $0.18^{***}$  \\
                            & $(0.00)$      & $(0.01)$      & $(0.01)$      & $(0.01)$      & $(0.01)$      & $(0.01)$      & $(0.01)$      & $(0.00)$      & $(0.00)$      \\
\midrule
Products in Market          &               & X             & X             & X             & X             & X             & X             & X             & X             \\
Gas Instruments             &               & X             &               &               & X             &               & X             &               & X             \\
Hub Interactions            &               &               & X             & X             & X             & X             & X             & X             & X             \\
Gandhi Instruments          &               &               &               &               &               & X             & X             & X             & X             \\
Exog Interactions           &               &               &               &               &               &               &               & X             & X             \\
Price Test                  &               & 101307.015    & 94273.624     & 94273.624     & 87736.444     & 94159.23      & 87480.095     & 78369.648     & 77831.661     \\
p-Value                     &               & 0             & 0             & 0             & 0             & 0             & 0             & 0             & 0             \\
Test of Over Identification &               & N/A           & 4187.14       & 4187.14       & 7279.16       & 4283.57       & 7526.31       & 11166.96      & 11605.17      \\
p-value                     &               & N/A           & 0             & 0             & 0             & 0             & 0             & 0             & 0             \\
R-Squared                   & $0.66$        & $-0.96$       & $0.32$        & $0.32$        & $0.27$        & $0.31$        & $0.28$        & $0.27$        & $0.28$        \\
Adj. R-Squared              & $0.66$        & $-0.96$       & $0.32$        & $0.32$        & $0.27$        & $0.31$        & $0.28$        & $0.27$        & $0.28$        \\
Mean Elasticity             & $-0.99$       & $-11.62$      & $-0.42$       & $-0.42$       & $-4.96$       & $-0.32$       & $-4.88$       & $-5.32$       & $-5.21$       \\
Median Elasticity           & $-1.00$       & $-11.74$      & $-0.42$       & $-0.42$       & $-5.01$       & $-0.33$       & $-4.93$       & $-5.37$       & $-5.27$       \\
Share Inelastic Products    & $0.51$        & $0.00$        & $1.00$        & $1.00$        & $0.00$        & $1.00$        & $0.00$        & $0.00$        & $0.00$        \\
Share JB Inelastic Products & $0.67$        & $0.00$        & $1.00$        & $1.00$        & $0.00$        & $1.00$        & $0.00$        & $0.00$        & $0.00$        \\
Share SP Inelastic Products & $0.88$        & $0.00$        & $1.00$        & $1.00$        & $0.00$        & $1.00$        & $0.00$        & $0.00$        & $0.00$        \\
Num. obs.                   & $307289$      & $307289$      & $307289$      & $307289$      & $307289$      & $307289$      & $307289$      & $307289$      & $307289$      \\
\bottomrule
\multicolumn{10}{l}{\scriptsize{$^{***}p<0.001$; $^{**}p<0.01$; $^{*}p<0.05$}}
\end{tabular}
        }
    \end{table}

\begin{table}
    \caption{Post-Pandemic Instrument Comparison Table}
    \label{tab:PostPand_Instrument_Compare}
    \resizebox{\linewidth}{!}{%
\begin{tabular}{l c c c c c c c c c}
\toprule
 & Model 1 & Model 2 & Model 3 & Model 4 & Model 5 & Model 6 & Model 7 & Model 8 & Model 9 \\
\midrule
Price                       & $-0.24^{***}$ & $-0.06$       & $1.50^{***}$  & $1.50^{***}$  & $0.01$        & $1.28^{***}$  & $-0.00$       & $-2.46^{***}$ & $-0.63^{***}$ \\
                            & $(0.00)$      & $(0.03)$      & $(0.16)$      & $(0.16)$      & $(0.03)$      & $(0.15)$      & $(0.03)$      & $(0.07)$      & $(0.03)$      \\
Nesting                     & $0.54^{***}$  & $-0.11^{***}$ & $-0.26^{***}$ & $-0.26^{***}$ & $-0.11^{***}$ & $-0.24^{***}$ & $-0.11^{***}$ & $0.15^{***}$  & $0.01^{**}$   \\
                            & $(0.00)$      & $(0.00)$      & $(0.01)$      & $(0.01)$      & $(0.00)$      & $(0.01)$      & $(0.00)$      & $(0.01)$      & $(0.00)$      \\
\midrule
Products in Market          &               & X             & X             & X             & X             & X             & X             & X             & X             \\
Gas Instruments             &               & X             &               &               & X             &               & X             &               & X             \\
Hub Interactions            &               &               & X             & X             & X             & X             & X             & X             & X             \\
Gandhi Instruments          &               &               &               &               &               & X             & X             & X             & X             \\
Exog Interactions           &               &               &               &               &               &               &               & X             & X             \\
Price Test                  &               & 77288.49      & 81853.459     & 81853.459     & 76910.107     & 81743.441     & 76883.509     & 66904.122     & 64536.553     \\
p-Value                     &               & 0             & 0             & 0             & 0             & 0             & 0             & 0             & 0             \\
Test of Over Identification &               & N/A           & 1269.2        & 1269.2        & 4165.87       & 1453.45       & 4271.75       & 7094.09       & 11604.74      \\
p-value                     &               & N/A           & 0             & 0             & 0             & 0             & 0             & 0             & 0             \\
R-Squared                   & $0.65$        & $0.34$        & $0.02$        & $0.02$        & $0.34$        & $0.09$        & $0.34$        & $0.16$        & $0.42$        \\
Adj. R-Squared              & $0.65$        & $0.34$        & $0.02$        & $0.02$        & $0.34$        & $0.08$        & $0.34$        & $0.16$        & $0.42$        \\
Mean Elasticity             & $-0.50$       & $-0.14$       & $3.19$        & $3.19$        & $0.02$        & $2.72$        & $0.00$        & $-5.22$       & $-1.34$       \\
Median Elasticity           & $-0.50$       & $-0.13$       & $3.14$        & $3.14$        & $0.02$        & $2.69$        & $0.00$        & $-5.15$       & $-1.33$       \\
Share Inelastic Products    & $1.00$        & $1.00$        & $0.02$        & $0.02$        & $1.00$        & $0.04$        & $1.00$        & $0.00$        & $0.23$        \\
Share JB Inelastic Products & $1.00$        & $1.00$        & $0.00$        & $0.00$        & $1.00$        & $0.01$        & $1.00$        & $0.00$        & $0.30$        \\
Share SP Inelastic Products & $1.00$        & $1.00$        & $0.11$        & $0.11$        & $1.00$        & $0.22$        & $1.00$        & $0.00$        & $0.84$        \\
Num. obs.                   & $265196$      & $265196$      & $265196$      & $265196$      & $265196$      & $265196$      & $265196$      & $265196$      & $265196$      \\
\bottomrule
\multicolumn{10}{l}{\scriptsize{$^{***}p<0.001$; $^{**}p<0.01$; $^{*}p<0.05$}}
\end{tabular}
    }
\end{table}

    \begin{figure}
        \caption{Distribution in Changes of Average Market Fare - Pre-Pandemic}
        \label{fig:AverageFare_ChangeDist_PrePandemic}
        \begin{center}
        \includegraphics[width = \linewidth]{PrePandemic_Merger_Change_AverageFare_Dist.pdf}
        \end{center}
    \end{figure}

    \begin{figure}
        \caption{Distribution in Changes of Average Market Fare - Post-Pandemic}
        \label{fig:AverageFare_ChangeDist_PostPandemic}
        \begin{center}
        \includegraphics[width = \linewidth]{Merger_Change_AverageFare_Dist.pdf}
        \end{center}
    \end{figure}
    
    \begin{table}
        \caption{Consistency in Estimated Minimum Fare Change Across Time}
        \label{tab:Merger_ConsistencyTable}
          \vspace{-15mm}
        \begin{center}
        
\begin{tabular}[t]{lrrr}
\toprule
Quarters & Best Case & Average Case & Worst Case\\
\midrule
\addlinespace[0.3em]
\multicolumn{4}{l}{\textbf{Pre-Pandemic}}\\
\hspace{1em}1 & 21 & 50 & 61\\
\hspace{1em}2 & 8 & 21 & 19\\
\hspace{1em}3 & 4 & 12 & 16\\
\hspace{1em}4 & 0 & 10 & 13\\
\hspace{1em}5+ & 4 & 11 & 42\\
\addlinespace[0.3em]
\multicolumn{4}{l}{\textbf{Post-Pandemic}}\\
\hspace{1em}1 & 32 & 81 & 93\\
\hspace{1em}2 & 11 & 31 & 27\\
\hspace{1em}3 & 3 & 21 & 23\\
\hspace{1em}4 & 1 & 19 & 26\\
\hspace{1em}5+ & 1 & 6 & 37\\
\bottomrule
\end{tabular}

        \end{center}
        \vspace{-5mm}
        \footnotesize{For each route with an estimated increase in minimum fare greater than \$40 estimated in at least one quarter, the number of quarters is reported for which that route is estimated to have an increase in minimum fare greater than \$40. Table \ref{tab:MinimumPrice} details distribution of minimum price changes in terms of markets.}
    \end{table}
    
\end{appendices}
	
\end{document}