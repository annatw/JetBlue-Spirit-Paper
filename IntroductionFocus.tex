\documentclass{article}
\usepackage{setspace}
\usepackage{natbib}

\usepackage{xr}
\externaldocument{main}

% Current Draft: May 12, 2025
\doublespacing
\begin{document}
    \section{Introduction}

    Horizontal mergers between two firms operating within the same markets reduce competitive pressures through the realization of complete coordination in pricing and quantity decisions between two formerly independent firms \citep{stigler_theory_1964}. However, they additionally allow for the movement of assets to higher value uses within the economy and the realization of economies of scale, potentially creating efficiencies sufficient enough to counteract these pressures \citep{williamson_economies_1968, kaplow_improving_2025}. As such, these mergers are able to increase or reduce both total welfare and consumer welfare while increasing profits for the merged firms \citep{farrell_horizontal_1990}. Globally, antitrust regulators generally move to block mergers that they view as likely to substantially lessen consumer welfare \citep{whinston_chapter_2007}.

    That the consumer welfare standard is standard globally does not mean that it is without its critics. Over the course of the last decade, the ``Neo-Brandeisian" movement has emerged who have argued that the move towards the consumer welfare standard has been misplaced, and that the 

    To understand these developments, I turn my attention towards the attempted merger between JetBlue Airways and Spirit Airlines. In 2022, JetBlue intervened with an unsolicited offer for Spirit to disrupt an attempted merger between Spirit and Frontier. The United States Department of Justice filed suit to block the JetBlue-Spirit merger the following year and would successfully block the merger following trial. 

    To examine this merger, I simulate the counterfactual world in which the merger had been completed through three different simulations, which are used as a best case, a worst case, and an ``average" case for realized efficiencies from the merger. Across my simulations, I find significant heterogeneity in the effects of the merger on markets that both firms operated within. 

    It is worth noting that the fares offered by Spirit Airlines are typically among the cheapest within the markets that they operate in. In line with this, I examine the change in the minimum fare available within JetBlue-Spirit markets within my simulations. I estimate that in the best-case merger simulation, over 35 markets would have the minimum fare increase within them by over \$60 had the merger been approved. Under the worst-case scenario assumptions, this number increases to over 200 markets. 

    With these findings outlined, the rest of the paper is organized as follows: Section \ref{sec:Literature} briefly summarizes the literature on airline mergers and role of low-cost carriers and ultra-low cost carriers within the industry; Section \ref{sec:Setting} details the relevant factors of the consumer aviation market within the United States; Section \ref{sec:Data} elaborates on my data sources and presents summary statistics; Section \ref{sec:Analysis} contains my analysis of the JetBlue-Spirit merger; finally, Section \ref{sec:Conclusion} summarizes the findings of this paper and examines their implications for antitrust policy going forwards. 

    
    \pagebreak 
	\bibliography{airline} 
	\bibliographystyle{abbrvnat.bst}

\end{document}