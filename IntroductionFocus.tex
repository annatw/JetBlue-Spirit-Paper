\documentclass{article}
\usepackage{setspace}
\usepackage{natbib}

\usepackage{xr}
\externaldocument{main}

% Current Draft: May 12, 2025
\doublespacing
\begin{document}
    \section{Introduction}

        Through merging, firms within the same industry are able to create a more efficient firm through the realization of economies of scale, shifting of assets to more productive uses, and sharing of productive technology between firms \citep{williamson_economies_1968, farrell_horizontal_1990, kaplow_improving_2025}. However, mergers necessarily require for two formerly independent firms to move towards a state of complete coordination which reduces competitive pressures within markets \citep{stigler_theory_1964}. Within the last fifty years, virtually all anti-trust regulators around the world have moved towards a ``consumer welfare" standard for approving mergers: if a merger represents a large reduction in competition, then to be approved it requires efficiencies sufficient to offset the motivation to increase prices \citep{whinston_chapter_2007}. This standard has come under attack by the ``Neo-Brandesian" movement of legal experts within the last decade, who believe that effective antitrust policy requires the movement to more restrictive merger approval regimes. During the Biden Administration, various members of this movement would be elevated to positions of power, including Lina Khan, as head of the Federal Trade Commission, and Jonathan Kanter, as head of the Department of Justice's antitrust division.  


    My paper examines the counterfactual pricing effects of the JetBlue-Spirit merger to examine if the merger would have negatively impacted consumers. % MY Question

    As the JetBlue-Spirit merger was never completed, I first estimate a structural model of supply and demand. For demand, I estimate a random coefficient nested logit model. Using these models, I recover product level marginal costs through a simple model of supply and then use these to simulate the effects of the JetBlue-Spirit merger across three separate specifications. To check the robustness of my results to the impact of the concurrent Northeast Alliance between JetBlue and American Airlines, I conduct these analyses on a sample consisting of the three years prior to the Covid-19 pandemic and on a sample consisting of the two years following the resumption of mass air travel in 2021. 

    I estimate that had the JetBlue-Spirit merger gone forward... It is worth noting that the fares offered by Spirit Airlines are typically among the cheapest within the markets that they operate in. In line with this, I examine the change in the minimum fare available within JetBlue-Spirit markets within my simulations. I estimate that in the best-case merger simulation, over 35 markets would have the minimum fare increase within them by over \$60 had the merger been approved. Under the worst-case scenario assumptions, this number increases to over 200 markets. 

    % Paragraph 2 - Different Merger Focus

    The existing literature is divided on the net price effects of the wave of mergers that has occurred within the aviation industry since the turn of the century. This division is due to differences in both merger characteristics and study methodology. To understand the impact of different methodologies on the estimated pro- or anti-competitive price effects, consider \citet{luo_price_2014} and \citet{carlton_are_2019}. These papers estimate pro-competitive effects of the United-Continental merger by using a differences-in-differences inference design which relies on the determination of pre- and post-periods. By contrast, \citet{fan_when_2020} finds evidence for price increases following this merger through the adoption of a model which allowed for dynamic price effects. The dynamic price effects model provides  evidence consistent with prices rising following key merger milestones other than completions. Research, such as \citet{bet_retrospective_2021, ciliberto_market_2021}, using structural modeling to recover firm marginal costs and markups to analyze the effects of these mergers have found similarly mixed evidence for the effects of mergers. \citet{bet_retrospective_2021} found evidence for increases in markups resulting from the United-Continental, Southwest-AirTran, and American-US Airways mergers but not the Delta-Northwest merger due to limited efficiency gains across these mergers. \citet{ciliberto_market_2021} estimated the effects of the American-US Airways merger using a structural model allowing for firms to reposition their product offerings post mergers. Through this model, they estimate price increases of around 5\% for duopoly markets consolidated by the merger. % My model is structural

    Finally, through the use of more recent data I examine the the evolution of the aviation industry through the period following the Covid-19 pandemic. I find that despite the decline in business travel caused by the pandemic, demand only became slightly more elastic post-pandemic, consistent with leisure travelers becoming less cost sensitive. Furthermore, I document changes in the role that low-cost carriers and ultra-low cost carriers play within the industry. Historically, low-cost carriers and ultra-low cost carriers have only rarely competed in the same markets \citep{ciliberto_market_2021}. Despite this, there would be a market with more than two carriers for every two markets with only a single low-cost carrier within it.  

    % Should mention sample periods

    With these findings outlined, I will briefly discuss the structure of the remainder of the paper. Section \ref{sec:Setting} details the relevant features of the consumer aviation market, as well as the proposed JetBlue-Spirit merger. Section \ref{sec:Data} elaborates on my data sources and presents summary statistics. Section \ref{sec:Analysis} contains my analysis of the JetBlue-Spirit merger and includes separate subsections for the demand model (section \ref{sec:Analysis_Demand}), supply model (section \ref{sec:Analysis_Supply}), and my simulation results (section \ref{sec:Analysis_Merger}).  I briefly conclude my paper in section \ref{sec:Conclusion} with a brief discussion of the findings of this paper and examination of their implications for antitrust policy going forwards. 

    
    \pagebreak 
	\bibliography{airline} 
	\bibliographystyle{abbrvnat.bst}

\end{document}


    To understand these developments, I turn my attention towards the attempted merger between JetBlue Airways and Spirit Airlines. In 2022, JetBlue intervened with an unsolicited offer for Spirit to disrupt an attempted merger between Spirit and Frontier. The United States Department of Justice filed suit to block the JetBlue-Spirit merger the following year and would successfully block the merger following trial. 

    In this paper, I estimate the counterfactual effects of the merger to understand if the merger would have raised prices had it been allowed to be completed. 
    
    To examine this merger, I simulate the counterfactual world in which the merger had been completed through three different simulations, which are used as a best case, a worst case, and an ``average" case for realized efficiencies from the merger. Across my simulations, I find significant heterogeneity in the effects of the merger on markets that both firms operated within. 

    It is worth noting that the fares offered by Spirit Airlines are typically among the cheapest within the markets that they operate in. In line with this, I examine the change in the minimum fare available within JetBlue-Spirit markets within my simulations. I estimate that in the best-case merger simulation, over 35 markets would have the minimum fare increase within them by over \$60 had the merger been approved. Under the worst-case scenario assumptions, this number increases to over 200 markets. 

        This paper analyzes the counterfactual effects of the proposed JetBlue-Spirit merger. To accomplish this, it draws from and contributes to the economic literature analyzing the merger-induced consolidation within the aviation industry during the twenty-first century.  It additionally furthers the understanding of the role of low-cost carriers and ultra-low cost carriers within the aviation industry in promoting competition through the analysis of more recent data which includes the period following the COVID-19 pandemic.
	
	Since the turn of the century, mergers have led to increased consolidation within the aviation industry. The existing literature is divided on the net price effects of these mergers, due to differences of merger characteristics and methodology. To understand the impact of different methodologies on the estimated pro- or anti-competitive price effects, consider \citet{luo_price_2014} and \citet{carlton_are_2019}. These papers estimated pro-competitive effects of the Delta-Northwest, United-Continental, and American-US Airways mergers by using a differences-in-differences inference design which did not allow for dynamic realization of effects and instead relied on the determination of pre- and post-period. This methodology would be challenged in \citet{fan_when_2020} which found evidence for price increases following the United-Continental merger through the adoption of a model which allowed for dynamic price effects. This found evidence consistent with prices rising following merger announcements rather than completion, which reduced the estimated effects in the prior literature. Research, such as \citet{bet_retrospective_2021, ciliberto_market_2021}, has additionally been conducted using structural modeling to recover firm marginal costs and markups to analyze the effects of these mergers. \citet{bet_retrospective_2021} found evidence for increases in markups resulting from the United-Continental, Southwest-AirTran, and American-US Airways mergers but not the Delta-Northwest merger while estimating limited efficiency gains across all of these mergers. \citet{ciliberto_market_2021} estimated the effects of the American-US Airways merger using a structural model which allowed for dynamic entry into markets, and estimated price increases of around 5\% for duopoly markets consolidated by the merger. Within this paper, structural modeling is used to assess the counterfactual effects of the JetBlue-Spirit merger. 

    In contrast to these papers focused on mergers of legacy carriers, this paper focuses on a proposed (but ultimately uncompleted) merger between a low-cost and an ultra-low cost carrier. As part of this analysis, it simulates the merger of these two airlines and finds evidence of heterogeneous fare increases following the merger. Recently, \citet{ciliberto_market_2021} and \citet{li_repositioning_2022} have used simulations of legacy carrier mergers to examine the implications of models that account for route entry decisions. I discuss the ramifications of these papers on the assumption of exogenous market structure for this paper in detail in Section \ref{sec:Analysis}. 

    Past research has identified strong competitive effects from low-cost carriers, such as Southwest, on incumbent firms within a market \citep{morrison_actual_2001, goolsbee_how_2008}. Spirit in particular has been linked to increased fare dispersion following its entry into a given route through inducing competing airlines to introduce 'basic economy fares'. This is associated with higher declines in price for cheaper fares than for more expensive fares, which Spirit less directly competes against \citep{shrago_spirit_2024}.

    % Above - Introduce Delta Anecdote?
    
	Finally, my paper touches upon the literature examining the role of low-cost and ultra-low cost carriers within the aviation industry. While this literature has predominantly focused on the ability of Southwest, the largest low-cost carrier, to lower prices within a market (e.g. \citet{windle_short_1995, morrison_actual_2001,  goolsbee_how_2008}), there has been movement in recent years to examine the effects of other low-cost carriers, such as JetBlue and Spirit, on prices. One example of this strand of literature is \citet{shrago_spirit_2024} which found that Spirit entry into a market was responsible for lower prices for the cheapest fares but minimal effect on average fares. In addition to my paper's analysis of the pricing effects of an unrealized merger between an ultra-low cost and low-cost carrier, it reports stylized facts about the changing role of low-cost and ultra-low cost carriers in the late 2010s and early 2020s.  


    With these findings outlined, the rest of the paper is organized as follows: Section \ref{sec:Setting} details the relevant factors of the consumer aviation market within the United States; Section \ref{sec:Data} elaborates on my data sources and presents summary statistics; Section \ref{sec:Analysis} contains my analysis of the JetBlue-Spirit merger; finally, Section \ref{sec:Conclusion} summarizes the findings of this paper and examines their implications for antitrust policy going forwards. 
