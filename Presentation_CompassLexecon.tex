\documentclass[xcolor=dvipsnames]{beamer}

\usepackage{setspace}
\usepackage{amsmath}
\usepackage{float}
\usepackage{longtable}
\usepackage{booktabs}
\usepackage{lscape}
\usepackage{graphicx}
\usepackage{silence}
\usepackage{forest}
\usepackage{hyperref}
\usepackage{placeins}
\usepackage{xcolor}
\usepackage{textcomp} % Needed on Windows in Office
\usepackage{derivative}
\usepackage{adjustbox}
\usepackage{tabularx}

\usepackage[style=apa]{biblatex}

\makeatletter
\g@addto@macro\normalsize{%
	\setlength\abovedisplayskip{0pt}
	\setlength\belowdisplayskip{-0pt}
}


\linespread{1.5}\selectfont

\usepackage[toc,page]{appendix}


\let\Oldsubsection\subsection
\renewcommand{\subsection}{\FloatBarrier\Oldsubsection}

\graphicspath{{05.Figures/}}

\bibliography{airline}

\author{Ann Atwater}
\institute{University of Florida}

\title{Distributional Effects of Mergers}
\subtitle{Evidence from Low-Cost Carriers}
\date{January 8, 2026}

\usetheme{Madrid}
\definecolor{UFOrange}{HTML}{FA4616}
\colorlet{beamer@blendedblue}{UFOrange}



\begin{document}
	\section{Introduction}
	\frame{\titlepage}
        \begin{frame}
        \frametitle{Motivation}
        \begin{itemize}
            \item Horizontal mergers simultaneously possess pro-consumer and anti-consumer aspects.
            \begin{enumerate}
                \item Pro-Consumer: Shifting of assets from less productive uses to more productive uses, realization of economies of scale create more effective, competitive firms (\cite{williamson_economies_1968, farrell_horizontal_1990, kaplow_improving_2025})
                \item Anti-Consumer: Two former rivals operate in a state of perfect collusion, creating lower competition. (\cite{stigler_theory_1964})
            \end{enumerate}
            \item Globally, antitrust regulators focus on consumer welfare to guide merger approval decisions. 
        \end{itemize}
    \end{frame}

    \begin{frame}
        \frametitle{Motivation}
        \begin{itemize}
         \item Throughout the economy, firms are heterogeneous in their customer bases.
            \begin{enumerate}
                \item Consumer Preferences: Vegan, Fast-Food, Steakhouses 
                \item Consumer Willingness to Pay: Legacy, Low-Cost, Ultra-Low Cost
            \end{enumerate}
        \item How does a merger of two firms which target different consumer segments impact consumer welfare?
        \begin{itemize}
            \item Is overall consumer welfare sufficient to reach a ``correct" decision?
        \end{itemize}
        \end{itemize}
         
    \end{frame}

    \begin{frame}
        \frametitle{This Paper}
        \begin{itemize}
            \item Analyzes These Questions Using Blocked JetBlue-Spirit Merger
            \begin{itemize}
                \item Estimates Structural Supply-Demand Model
                \item Simulates Merger Across Three Counterfactuals
                \begin{itemize}
                \item "Worst Case", "Average Case", "Best Case" scenarios for Approval
                \end{itemize}   
                \item Measures Analyzed
                \begin{itemize}
                    \item Minimum Fares 
                    \item Average Fares
                    \item Consumer Welfare
                \end{itemize}
            \end{itemize}
        \end{itemize}
    \end{frame}

    \begin{frame}
        \frametitle{Preview of Results}
        \begin{itemize}
                \item In JetBlue-Spirit Markets:
                \begin{itemize}
                \item Minimal Changes in Average Market Fares
                \item Merger Would Have Increased Consumer Surplus
                \begin{itemize}
                    \item Intuition: Superior JetBlue Products
                \end{itemize}
                \item Between 54 and 407 Markets Would Have Had Minimum Fares Increase by Over 50\%    
                \end{itemize}
                \item What if Merger Happened Pre-Pandemic?
                \begin{itemize}
                    \item Measures are Uniformly Consistent with Consumer Harm: Lower Surplus, Higher Minimum Fares
                \end{itemize}
        \end{itemize}
    \end{frame}
    
	\begin{frame}
		\frametitle{Presentation Organization}
		\begin{itemize}
				\item Setting
				\item Data and Summary Statistics
				\item Model and Results 
				\begin{itemize}
					\item Demand Model
                    \item Supply Model
                    \item Merger Simulations
                    \item Consumer Surplus
				\end{itemize}
                \item Conclusion
			\end{itemize}
	\end{frame}

    \section{Setting}
	\begin{frame}
		\frametitle{Types of Carrier}
		\begin{itemize}
			\item Three Types:
			\begin{itemize}
				\item Legacy: Delta, American, United
                \begin{itemize}
                    \item Hub-and-Spoke Networks
                \end{itemize}
				\item Low-Cost (JetBlue, Southwest)
				\begin{itemize}
					\item Focus on Direct Flights, Standardized Fleets
				\end{itemize}
				\item Ultra-Low Cost (Spirit, Allegiant, Frontier)
				\begin{itemize}
	            \item ``Unbundled Fares" 
                \item Cheaper Base Fares, Worse Quality    			\end{itemize}
			\end{itemize}
		\end{itemize}
	\end{frame}
	
    \begin{frame}
        \frametitle{JetBlue}
        \begin{itemize}
            \item Second largest Low-Cost Carrier in the United States
            \item Eastern Seaboard Network with Select Inland, West-Coast Markets
            \item Increasingly Anti-Competitive
            \begin{itemize}
                \item Northeast Alliance with American Airlines
            \end{itemize}
            \item Saw Purchasing Spirit As Way to Bolster Fleet
            \begin{itemize}
                \item Both Firms Primarily Use Airbus A320
                \item JetBlue Cancelled Aircraft Orders During 2020
            \end{itemize}
        \end{itemize}
    \end{frame}

    \begin{frame}
        \frametitle{Spirit}
        \begin{itemize}
            \item Largest Ultra-Low Cost Carrier in the United States
            \item Eastern Seaboard Network with Select Inland, West-Coast Markets
            \item Focus on Price Conscious Travelers
            \item Non-Fare ``Ancillary Fees" charged on ``Unbundled" tickets
            \begin{itemize}
                \item Checked Baggage Fees, Online Booking Fees, etc.
                \item These fees make up roughly half of its revenue from serving passengers 
            \end{itemize}
        \end{itemize}
    \end{frame}

    \begin{frame}
     \frametitle{Spirit - Revenue Sources over Time (10-K Filings)}
        \includegraphics[width = \linewidth]{05.Figures/Spirit_Revenue_Sources.pdf}
    \end{frame}


    \begin{frame}
        \frametitle{JetBlue-Spirit Merger Timeline}
        \begin{itemize}
        \item 2022:
        \begin{itemize}
            \item February: Frontier Attempts to Acquire Spirit
            \item April: JetBlue Intervenes with own Offer
            \item October: Spirit Shareholders Approve Merger with JetBlue
        \end{itemize}
        \item 2023: 
        \begin{itemize}
            \item March: DOJ, States File Suit
            \item October-December: Trial
        \end{itemize}
        \item 2024, January: Merger Blocked
        \begin{itemize}
            \item Ruling noted that JetBlue offers higher quality products than Spirit, but consumers "who must" rely on Spirit would be harmed by merger
        \end{itemize}
        \end{itemize}
    \end{frame}

    \begin{frame}
        \frametitle{Quarterly Ridership, All Carriers}
        \includegraphics[width = \linewidth]{Quarterly_DB1B_Itineraries}
    \end{frame}


    \begin{frame}
        \frametitle{Airline Industry After Covid-19}
        \begin{itemize}
        \item Airline Industry Was Dealing with Effects of Covid-19 During Attempted Merger
         \item Consumer Base of Industry Changed
            \begin{itemize}
                \item Business Travelers Historically Roughly Third of Air Travelers
                \begin{itemize}
                    \item Less Price Sensitive than Leisure Travelers
                \end{itemize}
                \item Business Travel Did Not Immediately Recover Post Pandemic
                \item Simultaneously: Leisure Travelers Built Up Cash Reserves
                \item Apriori Ambiguous Change of Demand Elasticity 
            \end{itemize}
            \item Three Years Pre-Pandemic Used as Additional, ``Normal" Sample for Comparison
        \end{itemize}
    \end{frame}

    \section{Data and Summary Statistics}

    \begin{frame}
		\frametitle{Data}
		\begin{itemize}
			\item Airline Origin and Destination Survey (DB1B)
			\begin{itemize}
				\item 10\% Sample of Domestic Passenger Itineraries 
				\item Data on Origin, Destination, Fare, Route, Carrier
                \item Ancillary Fees Not Included 
			\end{itemize}
		\end{itemize}
	\end{frame}


    \begin{frame}
        \frametitle{Market, Product Definitions}
        \begin{itemize}
        	\item Markets defined by Year-Quarter-Origin Airport-Destination Airport 
			\item Market Size is the Geometric Mean of the Population of Origin, Destination Metropolitan Statistical Areas
			\item Products Further Defined by Carrier, Nonstop Status
            \begin{itemize}
                \item In a Market, Firms can have 0, 1, or 2 Products
                \item Connecting Products have Characteristics Based on Weighted Average of Connecting Products
            \end{itemize}
        \end{itemize}
    \end{frame}

      \begin{frame}
        \frametitle{Summary Statistics - Product Level}
            \resizebox{\linewidth}{!}{%}

\begin{tabular}[t]{lllllll}
\toprule
\multicolumn{1}{c}{ } & \multicolumn{2}{c}{Pre-Pandemic} & \multicolumn{2}{c}{Post-Pandemic} & \multicolumn{2}{c}{ } \\
\cmidrule(l{3pt}r{3pt}){2-3} \cmidrule(l{3pt}r{3pt}){4-5}
 & Mean & (SD) & Mean & (SD) & Difference & t-Statistic\\
\midrule
Price (2017 USD) & 233.37 & (68.47) & 212.77 & (75.21) & 158.16 & 107.74***\\
Passengers & 4248.33 & (10185.27) & 3531.43 & (8648.27) & -4399.94 & 28.81***\\
Distance (1000s) & 1.41 & (0.67) & 1.41 & (0.67) & 0.74 & -0.11\\
Extra Distance & 0.13 & (0.18) & 0.14 & (0.19) & -0.06 & -12.61***\\
Nonstop & 0.28 & (0.45) & 0.26 & (0.44) & -0.16 & 10.51***\\
\addlinespace
Origin Destinations & 29.97 & (33.37) & 29.24 & (33.72) & -3.75 & 8.21***\\
Origin Presence (\%) & 36.23 & (31.26) & 34.77 & (30.92) & 5.31 & 17.73***\\
Observations & 307849 &  & 265196 &  &  & \\
\bottomrule
\end{tabular}

}
    \end{frame}

    \begin{frame}
        \frametitle{Summary Statistics - Market Level}
        \centering
    \resizebox{\linewidth}{!}{%}

\begin{tabular}[t]{lllllll}
\toprule
\multicolumn{1}{c}{ } & \multicolumn{2}{c}{Pre-Pandemic} & \multicolumn{2}{c}{Post-Pandemic} & \multicolumn{2}{c}{ } \\
\cmidrule(l{3pt}r{3pt}){2-3} \cmidrule(l{3pt}r{3pt}){4-5}
 & Mean & (SD) & Mean & (SD) & Difference & t-Statistic\\
\midrule
Minimum Miles & 1175.73 & (642.42) & 118.56 & (63.7) & 1057.17 & 483.44***\\
Mean Miles & 1228.56 & (661.79) & 124.25 & (65.7) & 1104.3 & 490.2***\\
Number of Firms & 2.94 & (1.49) & 3.21 & (1.56) & -0.27 & -34.91***\\
Number of Products & 3.52 & (2.11) & 3.79 & (2.16) & -0.26 & -24.29***\\
Number of Customers & 14970.24 & (28280.06) & 13375.81 & (25085.61) & 1594.43 & 11.84***\\
\addlinespace
HHI & 8017.21 & (4297.27) & 7479.76 & (4410.86) & 537.45 & 24.3***\\
\bottomrule
\end{tabular}

}
    \end{frame}

    \section{Model and Results}

    \begin{frame}
        \frametitle{Empirical Strategy}
        \begin{itemize}
            \item For Pre-Pandemic Period, Post-Pandemic Periods:
            \begin{itemize}
                \item Estimate Demand
                \item Impose Supply Assumption
                \item Recover Marginal Costs
                \item Estimate Merger Simulations
            \end{itemize}
            \item Why two periods? Post-Pandemic had
            \begin{itemize}
                \item Irregular Demand Patterns 
                \item Irregular Routing (Northeast Alliance)
            \end{itemize}
        \end{itemize}
    \end{frame}
    
    \subsection{Demand}
        \begin{frame}
        \frametitle{Demand Model}
        \begin{itemize}
            \item Random-Coefficient Nested Logit Model 
            \begin{itemize}
                \item Historically, Airline Demand has been estimated using nested logit model or the random-coefficient logit model.                \item Random coefficient nested-logit model outperforms other model types in simulations at estimating own-price and cross-price elasticities at increased computational costs (\cite{grigolon_nested_2014})
                \item Unlike some past research, use full spectrum of consumer types instead of two consumer types.
                \begin{itemize}
                    \item Spirit serves particularly price-sensitive travelers, not captured by dichotomy of two consumer types.  
                \end{itemize}
            \end{itemize}
            \item Two Nests: Air Travel, Outside Good
        \end{itemize}
    \end{frame}

    \begin{frame}
        \frametitle{Demand Model}
        \begin{itemize}
            \item  Consumer $i$ in market $t$ has indirect utility from buying product $j$ as defined by 
\[U_{ijt} = \delta_{jt} + \mu_{ijt} + \epsilon_{ijt}\]
        \item $\delta_{jt}$ is the mean utility across consumers in market $t$ for product $j$
		\item $\mu_{ijt}$ is the consumer's deviation from this mean utility
		\item  $\epsilon_{ijt}$ is an unobserved consumer-level shock
        \end{itemize}
    \end{frame}
    
    \begin{frame}
        \frametitle{Demand, Supply Estimation}
        \begin{itemize}
            \item Identification comes from four sets of instruments - Cost Shifters (Endpoint Hub Interactions), Interactions of Exogenous Regressors, Gandhi-Houde Differentiation Instruments, Number of Products in Each Market
            \item Consumers will purchase the good with the highest utility
            \item Utility of the outside good is normalized to zero, allowing for integration to recover estimated good shares.
            \item Supply Model: Bertrand Competition with Differentiated Products Following Exogenous Determination of Products
        \end{itemize}
    \end{frame}

    \begin{frame}
        \frametitle{Demand Model Results - Selected Coefficients}
        \tiny
        \centering
        \begin{adjustbox}{max height=\dimexpr\textheight-5.5cm\relax,
           max width=\textwidth}

\begin{tabular}[t]{lll}
\toprule
Variable & Pre-Pandemic & Post-Pandemic\\
\midrule
\addlinespace[0.3em]
\multicolumn{3}{l}{\textbf{Linear Coefficients}}\\
\hspace{1em}Price & -3.02*** & -3.11***\\
\hspace{1em} & (0.36) & (0.44)\\
\midrule
\addlinespace[0.3em]
\multicolumn{3}{l}{\textbf{Nonlinear Coefficients}}\\
\hspace{1em}Price & 0.592*** & 0.599***\\
\hspace{1em} & (0.12) & (0.12)\\
\midrule
\addlinespace[0.3em]
\multicolumn{3}{l}{\textbf{Nesting Coefficient}}\\
\hspace{1em}Nesting Parameter & 0.139*** & 0.115***\\
\hspace{1em} & (0.046) & (0.032)\\
\midrule
\addlinespace[0.3em]
\multicolumn{3}{l}{\textbf{Summary Statistics}}\\
\hspace{1em}Period & 2017Q1-2019Q4 & 2021Q2-2023Q2\\
\hspace{1em}N Products & 307849 & 265196\\
\hspace{1em}N Markets & 87363 & 70016\\
\bottomrule
\end{tabular}

\end{adjustbox}
    \end{frame}
    

    \begin{frame}
        \frametitle{Supply Model Results}
        \tiny
        \centering
        \begin{adjustbox}{max height=\dimexpr\textheight-5.5cm\relax,
           max width=\textwidth}

\begin{tabular}[t]{lll}
\toprule
Variable & Pre-Pandemic & Post-Pandemic\\
\midrule
\addlinespace[0.3em]
\multicolumn{3}{l}{\textbf{Summary Statistics}}\\
\hspace{1em}Period & 2017Q1-2019Q4 & 2021Q2-2023Q2\\
\hspace{1em}N Products & 307849 & 265196\\
\hspace{1em}N Markets & 87363 & 70016\\
\hspace{1em}Mean Elasticity & -5.519 & -5.211\\
\hspace{1em}Spirit Mean Elasticity & -4.07 & -3.44\\
\hspace{1em}JetBlue Mean Elasticity & -5.34 & -5.18\\
\hspace{1em}Mean Markup (\%) & 19.38 & 20.97\\
\bottomrule
\end{tabular}

\end{adjustbox}
    \end{frame}

    \subsection{Merger Simulation}
   	\begin{frame}
		\frametitle{Merger Simulation}
			\begin{itemize}
            \item Estimate three counterfactuals for each period.   
            \begin{table}
            \centering
            \begin{tabular}{c|c|c}
                Counterfactual & Marginal Cost & Unobservables\\\hline 
                Best Case & Minimum & Maximum \\
                Average Case & Average & Average\\
                Worst Case & Maximum & Minimum \\
            \end{tabular}
        \end{table}

       \item Like products get combined with like products: Nonstop with Nonstop, Connecting with Connecting
                    \begin{itemize}
                        \item Merged Connecting Products Have Minimum Miles 
                        \item All products have JetBlue Fixed Effects
                    \end{itemize}
        \end{itemize}
	\end{frame}

    \begin{frame}
        \frametitle{Why Care about Minimum Market Fare?}
        \begin{itemize}
            \item In the judgment blocking the merger, the core element of consumer harm was the existence of consumers who'd exit the market without Spirit's unbundled fares being offered.
            \item By analyzing the change in the simulated minimum fare after the merger, we can gain an understanding of how realistic this issue is.
            \item Today, this is the first of two merger simulation results that I present.
        \end{itemize}
    \end{frame}

    \begin{frame}
        \frametitle{Merger Results - Percent Change Minimum Market Fare}
    \vspace{-12mm}
   \begin{table}
   \resizebox{0.9\linewidth}{!}{%}
        \begin{tabular}[t]{lrrrrrr}
\toprule
\multicolumn{1}{c}{ } & \multicolumn{3}{c}{Pre-Pandemic} & \multicolumn{3}{c}{Post-Pandemic} \\
\cmidrule(l{3pt}r{3pt}){2-4} \cmidrule(l{3pt}r{3pt}){5-7}
 & Best & Average & Worst & Best & Average & Worst\\
\midrule
$<$ 0 & 256 & 206 & 167 & 389 & 302 & 230\\
0-25 & 1186 & 830 & 710 & 1058 & 650 & 617\\
25-50 & 56 & 419 & 383 & 53 & 414 & 300\\
50-75 & 19 & 62 & 206 & 28 & 137 & 246\\
75-100 & 11 & 11 & 56 & 16 & 36 & 109\\
100 $<$ & 5 & 5 & 11 & 10 & 15 & 52\\
\bottomrule
\end{tabular}
    }
    \end{table}
    \end{frame}


    \begin{frame}
        \frametitle{Consumer Surplus}
        \begin{itemize}
            \item Calculating consumer surplus with my data is hindered by the lack of data on ancillary fees.\begin{itemize}
            \item A customer who will pay for checked bags when flying with Spirit will have her change in consumer surplus over-estimated.    
            \end{itemize}
            \item To better understand how the lack of data on ancillary fees impacts my estimates, I adjust observed prices using data from Spirit's annual 10-K financial filings.
            \item As such, I scale the average ancillary fee charged by Spirit by overall market fare and add it to the recorded fare.
            \begin{itemize}
                \item Then re-estimate model.
            \end{itemize}
        \end{itemize}
    \end{frame}

    \begin{frame}
        \frametitle{Change in Consumer Surplus Estimates}
        \tiny
        \centering
\resizebox{0.75\linewidth}{!}{%
    \begin{tabular}[t]{lrr}
\toprule
& \multicolumn{2}{c}{\textbf{Total Change in Consumer Surplus}} \\
\cmidrule(lr){2-3}
& \textbf{Pre-Pandemic} & \textbf{Post-Pandemic} \\
\midrule
\addlinespace[0.3em]
\multicolumn{3}{l}{\textbf{Main Merger Analysis}} \\
\addlinespace[0.1em]
\hspace{1em}Best Case & 3,740,089,657 & 6,943,764,751 \\
\hspace{1em}Average Case & -355,595,347 & 1,665,903,471 \\
\hspace{1em}Worst Case & -1,294,518,659 & 515,351,671 \\
\addlinespace[0.3em]
\multicolumn{3}{l}{\textbf{Ancillary Fee Adjustment Merger Analysis}} \\
\addlinespace[0.1em]
\hspace{1em}Best Case & -161,377,502 & 2,019,197,851 \\
\hspace{1em}Average Case & -1,174,813,263 & 765,481,251 \\
\hspace{1em}Worst Case & -1,500,841,809 & 379,642,080 \\
\bottomrule
\end{tabular}
}

    (Note: JetBlue-Spirit Markets Only)
    \end{frame}

    \begin{frame}
        \frametitle{Change in Consumer Surplus Findings}
        \begin{itemize}
            \item Change in consumer surplus is greater without the adjustment for ancillary fees
            \item However, merger is consistently estimated to be favorable (unfavorable) to consumer welfare in the post-pandemic (pre-pandemic) period.
            \begin{itemize}
                \item Increases of over \$350 million in even the worst case simulation for the post-pandemic period.
            \end{itemize}
        \end{itemize}
    \end{frame}

     % CONSUMER LEVEL DEVIATION FROM MEAN UTILITY
    
    \section{Conclusion}
    \begin{frame}
        \frametitle{Conclusion: On the Merger}
        \begin{itemize}
            \item Some consumers would have been plausibly forced out of the market.
            \begin{itemize}
                \item Over 35 markets had minimum fares increase by over 50\% in even the best-case simulation
            \end{itemize}
            \item However, in the post-pandemic period, merger would have increased consumer surplus within JetBlue-Spirit markets.
            \begin{itemize}
                \item Consumers prefer JetBlue products to Spirit products ceteris paribus.
            \end{itemize}
            \item However, under a cost-benefit analysis, proper approval of this merger would have depended on how likely the policy maker believes the post-pandemic travel environment is to continue.
            \begin{itemize}
                \item Policy makers that believe that consumer demand will return to pre-pandemic norms should always block the merger
            \end{itemize}
        \end{itemize}
    \end{frame} 

    \printbibliography
%	\bibliographystyle{abbrvnat.bst}

    \section{Clarifying Slides}
    

\end{document}


